\documentclass[ngerman]{article}
%pdf language settings
\usepackage[german]{babel}
\usepackage[utf8]{inputenc}
\usepackage[T1]{fontenc} %provides a better way for westeuropean characters and such 
\setlength{\parindent}{0cm} %noindent for all paragraphs

%pdf page settings
\usepackage[a4paper,top=30mm,bottom=40mm,inner=25mm,outer=35mm]{geometry} 
\usepackage{scrlayer-scrpage} %Headers and footers
%math environments
\usepackage{amssymb}
\usepackage{amsthm}
\usepackage{amsfonts}
\usepackage{amsmath}
\usepackage[nothm]{thmbox} %Decorate theorem statements
%more features for environments and other stuff
\usepackage{centernot} %nicer looking not
\usepackage{marvosym} %some symbols
\usepackage{paralist} %Adds more function to listing environments
\usepackage{hyperref}
%Headers and Footnotes
\pagestyle{headings}
\ihead{Funktionalanalysis - Lösungsvorschläge zu den Aufgaben}
\usepackage{stmaryrd}

\usepackage[explicit]{titlesec}%customize titles
%\renewcommand{\thesection}{{\Roman{section}}}
%\renewcommand{\thesubsection}{{\theenumi{subsection}}} 
\titleformat{\section}{\normalfont\Large\bfseries}{}{0em}{{\thesection}. #1}
%\titleformat{\section}{\normalfont\Large\bfseries}{\Roman{\thesection} . #1}{0em}{}
\titleformat{\subsection}{\normalfont\large\bfseries}{}{0em}{{\thesubsection}\, #1}  
%\titleformat{\subsection}{\normalsize\bfseries\itshape}{\textnormal{\roman{subsection}.}}{1em}{}

%mathematical environments
\theoremstyle{definition}%Lässt den Text innerhalb nicht mehr kursiv erscheinen
\newtheorem*{beh}{Behauptung}

%\renewcommand\thechapter{\Roman{chapter}}

%mathematical Macros
\newcommand{\C}{\mathbb{C}}
\newcommand{\R}{\mathbb{R}}
\newcommand{\pR}{\mathbb{R}_{\geq 0}} 
\newcommand{\Q}{\mathbb{Q}}
\newcommand{\Z}{\mathbb{Z}}
\newcommand{\N}{\mathbb{N}}
\newcommand{\K}{\mathbb{K}}

\newcommand{\hA}{\mathfrak{A}}%Script A , hA ~ hässliches A
\newcommand{\hL}{\mathcal{L}}
\newcommand{\tT}{\mathcal{T}} %topologisches T
\newcommand{\B}{\mathcal{B}} %Raum der beschränkten Funktion
\newcommand{\BS}[1][X,Y]{\mathcal{B}(#1)} %"standardraum" der beschränkten Funktion

\newcommand{\lL}[2][\Omega,\mu]{\text{L}^{#2}(#1)} %Lebesgue L für Matheumgebgung mit Omega, mu als Standard \lL[Raum,Maß]{Dimension}
\newcommand{\ess}{\textnormal{ess}}
\newcommand{\supp}{\textnormal{supp}}
 
%\newcommand{\lightning}{\text{\Lightning}}
\newcommand{\abs}[1]{\left\lvert #1 \right\rvert }
\newcommand{\norm}[1]{\left\lvert \left\lvert #1 \right\rvert \right\rvert}

\newcommand{\intl}[1]{\int_\Omega #1 d\mu} %"Standard"Lebesgue-Integral über Omega
\newcommand{\df}{\Rightarrow} % "daraus folgt", also Implikation
\newcommand{\U}[2][1]{U_{#1}(#2)} %Umgebung um #2 mit Abstand #1 \U{r}{a}
\newcommand{\EK}{\U{0}} %Umgebung bzw Einheitskugel um 0
\newcommand{\limes}[1][\infty]{\lim_{n \to #1}}
\newcommand{\afs}{"{}}
\newcommand{\im}{\textnormal{im }}
\newcommand{\aufspan}{\textnormal{span}}

\newcommand{\semi}[1]{\interleave{#1}\interleave}

\renewcommand{\{ }{\left\lbrace}
\renewcommand{\}}{\right\rbrace}

\begin{document}
\tableofcontents
\newpage
%%%%%%%%%%%%%%%%%%%%%%%%%%%%%%%%%%%%%%%%%%
%Blatt 1
%%%%%%%%%%%%%%%%%%%%%%%%%%%%%%%%%%%%%%%%%%

\section{Blatt 1}

%%%%%%%%%%%%%%%%%%%%%% Aufgabe 1
\subsection{Reihen und Vollständigkeit}
Es sei $(X, \|\cdot\|)$ ein normierter Raum.
\begin{beh}
Es ist äquivalent:
\begin{enumerate}[(a)]
	\item	$\sum^\infty_{n=0} \|u_n\|<\infty\df$ Die Reihe $\sum^\infty_{n=0}u_n$ konvergiert in $X$.
	\item	$X$ ist ein Banachraum.
\end{enumerate}
\end{beh}
\begin{proof}
	\afs a$\df$b\afs :	
	Da $X$ bereits ein normierter Raum ist, ist nur noch zu zeigen, dass $X$ vollständig ist. Sei $(u_n)$ eine Cauchyfolge in $X$, wobei o.B.d.A $u_0=0$, sonst betrachten wir $(\tilde{u}_n)$ mit $\tilde{u}_{n+1}:=u_n,\,\tilde{u}_0:=0$ statt $(u_n)$.
	Es ist $u_n = \sum^k_{n=0} u_n - u_{n-1}$. 
	Sei $x_n : = \frac{1}{n^2}$.	
	Da $u_n$ eine Cauchyfolge ist, gibt es für jedes $x_n$ ein (kleinstes) $N_n:=N(x_n)\in\N$, so dass $\| u_{N(x_n)}-u_m\| \leq x_n\; \forall m \geq N_n$.
	Das Bemerkenswerte an $(N_n)$ ist, dass diese Folge monoton steigt. 
	Es ist nun für $n'\in\N$ mit der Dreiecksungleichung und der Eigenschaft der Teleskopsumme
	$$\norm{ \sum^{N({n'+1})}_{n=0}u_n-u_{n-1}}\leq \underbrace{\|u_1 - u_0 \| + \| u_2 - u_1\| + \dots }_{=:x_0}+ \underbrace{\|u_{N_2} - u_{N_1}\|}_{\leq x_1} + \underbrace{\|u_{N_3} - u_{N_2}\|}_{\leq x_2} + \dots + \underbrace{\|u_{N(n'+1)}-u_{N(n')}\|}_{\leq x_{n'}}$$
	Bzw.
	$$ \norm{\sum^{N({n'+1})}_{n=N_2} u_{N_n} - u_{N_{n-1}} } \leq {\sum^{N({n'+1})}_{n=N_2} \norm{u_{N_n} - u_{N_{n-1}}} } \leq \sum^{n'}_{n=1} x_n$$
	
	%$$\|u_k\| \leq \sum^{n'}_{n=0}x_n \leq \sum^k_{n=0}x_n \df \lim_{k\to\infty} \|s_k\| \leq \sum^\infty_{n=0}x_n < \infty$$ 
	Damit erfüllt ${\sum^{N({n'+1})}_{n=N_2} \norm{u_{N_n} - u_{N_{n-1}}} }$ das Cauchykriterium für Reihen, und somit konvergiert sie. Es gilt also:
	$${\sum^{\infty}_{n=N_2} \norm{u_{N_n} - u_{N_{n-1}}} } < \infty.$$
	Mit der Eigenschaft a) ist nun: $u_{N_n} \overset{n\to\infty}{\longrightarrow} \alpha \in X$. Da $(u_k)$ eine Cauchyfolge ist, und sie eine konvergente Teilfolge hat, bleibt ihr nichts anderes übrig, als selber auch gegen $\alpha$ zu konvergieren. Also konvergiert $u_k$ in $X$, damit ist $X$ ein Banachraum.\par\medskip
  
	
 b$\df$a:
% 	Sei X ein Banachraum und $(u_n)$ eine Folge in $X$, so dass $\sum^\infty_{n=0}\|u_n\| < \infty$, dann ist dank der Dreiecksungleichung: $\| \sum^\infty_{n=0}u_n \| \leq  \sum^\infty_{n=0}\|u_n\| < \infty$.\\
% 	Dank der Vollständigkeit von $X$ können wir das Majorantenkriterium (der Beweis zu der Gültigkeit dieses Kriteriums, ist analog zum reellen Fall) anwenden, und somit konvergiert die Reihe $\sum^\infty_{n=0}u_n$.\par
	Sei X ein Banachraum und $(u_n)$ eine Folge in $X$, so dass $s:=\sum^\infty_{n=0}\|u_n\| < \infty$. Insbesondere erfüllt $s$ das Cauchykriterium für Reihen, und es ist dank der Dreiecksungleichung für $\varepsilon > 0$
	$$\norm{ \sum^m_{n=k}u_n} \leq  \sum^m_{n=k}\|u_n\| < \varepsilon \quad \forall m,k\geq N(\varepsilon).$$
 	ALso erfüllt auch $\sum^m_{n=k}u_n$ das Cauchykriterium für Reihen. Damit konvergiert sie in $X$, also ist (a) gezeigt.
 	
\end{proof}

%%%%%%%%%%%%%%%%%%%%%%%Aufgabe 2
\subsection{Die Operatornorm}
Es seien $(X,\|\cdot\|_X)$ und $(Y,\|\cdot\|_Y)$ normierte Räume und $T\in \B(X,Y)$.
\begin{beh}
	Es gilt:
	$$\|T\|:=\sup_{x\in U_1(0)} \|Tx\|_Y = \sup_{x\in\overline{U_1(0)}}\|Tx\|_Y = \sup_{x\in \partial U_1(0)} \|Tx\|_Y = \sup_{x\in X\backslash \{0\}} \frac{\|Tx\|_y}{\|x\|_X}.$$
\end{beh}
\begin{proof}
1.Variante: 
	Wir beweisen jede Gleichheit einzeln. Das erste Gleichheitszeichen ist eine Definition und somit wahr.\\
	\afs$\sup_{x\in U_1(0)} \|Tx\|_Y \leq \sup_{x\in\overline{U_1(0)}}\|Tx\|_Y$\afs :
	Da $U_1(0) \subseteq  \overline{U_1(0)}$ kann das Supremum nicht größer werden.\\
	\afs$\sup_{x\in U_1(0)} \|Tx\|_Y \geq \sup_{x\in\overline{U_1(0)}}\|Tx\|_Y$\afs:
	Es sei $(x_n)$ eine Folge aus $U_1(0)$ mit $x_n \to x \in \overline{U_1(0)}$. Es ist erstmal $\norm{ Tx_n}_Y \leq \norm { T }\; \forall n\in\N$. Wegen der Stetigkeit von $T$ (Satz 1.7) und $\norm{ \cdot }_Y$ ist $\norm{Tx_n}_Y \to \norm{Tx}_Y \df \norm{Tx}_Y \leq \norm{T}$.\\
	Für die restlichen Gleichheiten wird freundlich auf die 2. Variante verwiesen.\par\bigskip
		
2.Variante:
	Aus Satz 1.7 folgt, dass $T$ stetig ist. \par 
	Dank der Definition von $\sup$ und der Stetigkeit von $T,\|\cdot \|_Y$ folgt: $\displaystyle \|T\|=\sup_{x\in\overline{U_1(0)}}\|Tx\|_Y$ \par 
	Nun ist: 
	\begin{equation*}
\begin{split}
\sup_{x\in\overline{U_1(0)}}\|Tx\|_Y & = \sup_{x\in\overline{U_1(0)}\backslash\{0\}}\|\|x\|_X T\left(\frac{x}{\|x\|_X}\right)\|_Y
\\ & = \sup_{x\in\overline{U_1(0)}\backslash\{0\}}\|x\|_X \|T\left(\frac{x}{\|x\|_X}\right)\|_Y
\\ & = \underbrace{\sup_{x\in\overline{U_1(0)}\backslash\{0\}}\|x\|_X}_{=1}
	   \cdot \sup_{x\in\overline{U_1(0)}\backslash\{0\}} \|T\left(\frac{x}{\|x\|_X}\right)\|_Y
\\ & = \sup_{x\in \partial U_1(0)} \|Tx\|_Y  =
	\sup_{x\in X\backslash\{0\}} \|T\left(\frac{x}{\|x\|_X}\right)\|_Y
 = \sup_{x\in X\backslash\{0\}} \frac{\|Tx\|_Y}{\|x\|_X}
\end{split}
\end{equation*}
Wobei wir genuzt haben, dass $T0 = 0$ (für das erste Gleichheitszeichen) und $\|x\|_X,\|Tx\|_Y\geq 0\;\forall x\in X$ (für das dritte Gleichheitszeichen). 
\end{proof}

%%%%%%%%%%%%%%%%Aufgabe 3
\subsection{Eigenschaften in endlichdimensionalen Vektorräumen}
Es seien $(X, \| \cdot \|_X)$ und $(Y, \| \cdot \|_Y)$ normierte Räume.
\begin{enumerate}[(a)]
\item

\begin{beh}
$$m: = dim(X) < \infty \text{ und } T:X\to Y \text{ linear}\df T \in \B(X,Y)$$
\end{beh}
\begin{proof}
	Vorweg eine Vorüberlegung: 
	Es sei $B:=(v_1,\dots,v_m)$ eine Basis von $X$. Ist $x_n = \sum^m_{i=1} \alpha^n_i v_i,\, \alpha^n_i \in \K$ eine Folge aus $X$ mit $x_n \to x = \sum^m_{i=1} \alpha_i v_i$, so muss gelten $(\alpha^n_1,\dots\alpha^n_m) \overset{n\to\infty}{\longrightarrow} (\alpha_1,\dots,\alpha_m)$. Dies folgt, da im $\K^m$ alle Normen äquivalent sind, und im $\K^m$ komponentenweise Konvergenz äquivalent zur Normkonvergenz ist ($\nearrow$ Folgerung 4.29 im Ana I/II Lindner Skript).
	Falls einem dies nicht genügt, so bediene er sich der Seiten 102-103 im Buch Funktionalanalysis von H. Heuser 3.Auflage.	
	\par\bigskip
	1.Variante:
	Für den eigentlichen Beweis nutzen wir Satz 1.7 und zeigen, dass $T$ stetig in der Null ist. 
	Dafür sei $(x_n)$ eine Folge aus $X$ mit $\limes x_n = 0$ und $x_n = \sum^m_{i=1} \alpha^n_i v_i,\, \alpha^n_i \in \K$.
	$$\norm{ Tx_n}_Y = \norm{ T \sum^m_{i=1} \alpha^n_i v_i }_Y = \norm{ \sum^m_{i=1} \alpha^n_i Tv_i}_Y \leq \sum^m_{i=1} \abs{\alpha^n_i} \norm{Tv_i}_Y \leq \max^m_{i=1} \norm{T v_i} \sum^m_{i=1} \abs{\alpha^n_i} $$
	Wegen der Vorüberlegung ist 
	$$\limes \sum^m_{i=1} \abs{\alpha^n_i} = 0 \df \limes Tx_n = 0$$\par\bigskip

	2.Variante:
Da $dim(X) <\infty$ und $T$ linear ist, ist der Bildraum ein endlichdimensionaler Vektorraum, und es gibt eine Basis $C$ von $\im T$. Es gibt nun, dank der linearen Algebra eine Matrix $M \in \K^{m\times \dim \im T}$, so dass $T = \phi^{-1}_C \circ M \circ \phi_B$, wir zeigen nun, dass eine Koordinatenabbildung $\phi$ ein Homöomorphismus ist, denn dann ist $T$ eine Komposition aus stetigen Funktionen und somit selber stetig.
Wegen der Vorüberlegung ist $\phi: X \mapsto \K^m$ stetig. Offenbar ist $\phi$ bijektiv.  Da die Addition stetig ist, ist auch $\phi^{-1}$ stetig.
$M$ ist auch stetig denn 
$$\forall x\in \K^m,\,\norm{x} < 1 \text{ gilt } \|Mx\|\leq a\|Mx\|_{\infty} \leq a\|M\|_{\infty} \|x\| \leq a\|M\|_{\infty} < a\infty.$$
Wobei $a\in\R^+$ die geeignet gewählte nur von der Norm abhängige Konstante ist, die von der Äquivalenz zu jeder anderen Norm entsteht.
\end{proof}

\item

\begin{beh}
$$dim(X)<\infty \text{ und } T\in\B(X,Y) \df \|T\|= \max_{x\in\overline{U_1(0)}}\|Tx\|_Y$$
\end{beh}
\begin{proof}
Wir greifen vor auf Blatt 4 Aufgabe 3. Es ist nämlich eine Teilmenge eines endlichdimensionalen Vektorraumes genau dann kompakt, wenn sie abgeschlossen und beschränkt ist. Deshalb ist schon mal $\overline{U_1(0)}$ kompakt. Aus Aufgabe 2 ist bekannt, dass $\displaystyle\sup_{x\in U_1(0)} \|Tx\|_Y = \sup_{x\in\overline{U_1(0)}}\|Tx\|_Y$. $T,\,\|\cdot\|_Y$ sind dank Satz 1.7 stetig, damit auch $\norm{T \cdot}_Y$. Da $\overline{U_1(0)}$ kompakt ist und stetige Funktionen, die auf $\R$ abbilden, auf kompakten Mengen ihr Maximum annehmen, ist die Behauptung bewiesen.
\end{proof}

\item

\begin{beh}
Im allgemeinen ist die Aussagen von b) falsch.
\end{beh}
\begin{proof}
Wir geben ein Beispiel für $T\in\B(X,Y)$ derart an, dass 
$$\|T\| \not\in\left\lbrace\|Tx\|_Y|\,x\in\overline{U_1(0)}\right\rbrace.$$
1.Variante: Es sei $X = \aufspan (e_k)_{k\in\N} \subseteq \ell^1,\; Y = \R$, wobei $e_k = (\delta_{ik})_{i\in\N}$, und 
	$$T:X\to \R,\quad x \mapsto Tx := \sum^\infty_{k=1} \frac{1}{2^k}x_k$$
	$T$ ist offenbar linear und wegen $\abs{Tx} = \abs{ \sum^\infty_{k=1} \frac{1}{2^k}x_k } \leq \sum^\infty_{k=1} \abs{x_k} = \norm{x}_1$ auch beschränkt. Es ist 
	$$\overline{\left\lbrace x \in \ell^1 : x \in X, \norm{x} < 1 \right\rbrace }  = $$ TODO


2.Variante:
Sei  $$X=Y=\ell^p(\K)=:\ell^p \text{ (beschr. Folgenraum) und } T:\ell^p\to \ell^p,\;(x_n)\mapsto \left(\left(1-\frac{1}{n}\right)x_n\right)$$
Mit der Norm $\displaystyle \|(x_n)\|=\left(\sum^\infty_{n=1} |x_n|^p\right)^{\frac{1}{p}}$\par
Dann ist $\overline{U_1(0)} = \left\lbrace (x_n) \in \ell^p |\,\sum^\infty_{n=1} |x_n|^p \leq 1 \right\rbrace$\par
$T$ ist ein beschränkter linearer Operator, denn:
\begin{enumerate}[(i)]
\item Seien $(x_n),(y_n)\in \ell^p,\,\alpha \in \K\df T(\alpha x_n + y_n) = \left(\left(1-\frac{1}{n}\right)(\alpha x_n + y_n) \right) = \alpha Tx_n + Ty_n$

\item Sei $x_n \in \overline{U_1(0)}$
\begin{equation*}
\begin{split}
 \df 
 \|T(x_n)\|^p 
 = \|\left(\left(1-\frac{1}{n}\right)x_n\right)\|^p 
 & = \sum^\infty_{n=1} \left(\left(1-\frac{1}{n}\right)x_n\right)^p 
 \\ & = \sum^\infty_{n=1} | x_n-\frac{x_n}{n}|^p 
 \\ & \leq \sum^\infty_{n=1} |x_n|^p+|\frac{x_n}{n}|^p
 \\ & \leq \sum^\infty_{n=1} 2|x_n|^p < \infty
 \end{split}
\end{equation*}


\end{enumerate}
Mit der Folge (von Folgen) $i_n$, die an der $n.$ Stelle eine 1 hat, und sonst nur Nullen, ist dann: $T(i_n) \longrightarrow 1$ für $n\longrightarrow \infty$. Dies ist mit Elementen aus $\overline{U_1(0)}$ nicht möglich (Warum?).\par
Nachtrag:\\
Geschickter ist es den Folgenraum $M\subset \ell^p$ einzuschränken, so dass für jedes Element aus $M$ gilt, dass nur endlich viele Folgenglieder ungleich null sind. Der Rest folgt flott.


3.Variante mit einem Integraloperator.
\end{proof}

\end{enumerate}

%%%%%%%%%%%%%%%%%%%%%%%%%%%%%%%%%%%%%%%%%%
%Blatt 2
%%%%%%%%%%%%%%%%%%%%%%%%%%%%%%%%%%%%%%%%%%

\newpage
\section{Blatt 2}

%%%%%%%%%%%%%%%%%%%Aufgabe 1
\subsection{Beispiele}
\begin{enumerate}[a)]
\item $T_1 \in \B(X)$ injektiv mit $\overline{\im T_1} =X$, aber $\im T_1\not = X$.\par
	Die Eigenschaft bedeutet gerade, dass wir ein unter einem linearen injektiven Operator dichtes Bild haben wollen, dass nicht das ganze Bild umfasst.\par
	1.Variante: 
	Es sei $X := \ell^1$ und 
	$$T_1(x_1,x_2,x_3,\dots) := (x_1,\frac{1}{2}x_2, \frac{1}{3}x_3,\dots)$$
	$T_1$ ist offenbar linear und wegen $T_1x = 0 \df x = 0$ injektiv. Es ist 
	$$T_1n\cdot e_n = e_n \df \aufspan (e_k)_{k\in\N} \subseteq \im T_1.$$
	Wegen Aufgabe  2 ist also $\overline{\im T_1} = \ell^1$. Allerdings ist $\im T_1 \not = \ell^1$, denn zum Beispiel $(x_n) = (\frac{1}{n^2}) \not \in \im T_1$. 
	Es ist nämlich $T(\frac{1}{n}) = (\frac{1}{n^2})$, und wegen der Injektivität gibt es keine weiteren Elemente, die das erfüllen. Aber $(\frac{1}{n})\not \in \ell^1$. \par\bigskip
	2.Variante:
	Sei $X := \{f\in C[0,1] : f(0) = 0\}$ mit der Supremumsnorm und 
	$$T:X\to X,\quad\text{mit } f\in X,\, x\in[0,1],\, T(f)(x):=\int^x_0 f(t) dt.$$
	Es ist $\im T = \{f\in C^1[0,1] : f(0) = 0\}$, und dies liegt dank der höheren Analysis dicht in $X$, aber ist offenbar nicht ganz $X$.
%Dafür sei $C^k_0[0,1]:=\{f\in C^k[0,1] : f(0) = 0\},\;k\in\N_0$.
%Nun betrachten wir $(X,\|\cdot\|) = (C^0_0[0,1],\|\cdot\|_\infty)$ die Menge der stetigen Funktionen auf $[0,1]$ mit $f(0)=0$ versehen mit der Supremumsnorm, und 
%$$T:X\to X,\;\text{ mit } f\in X,\, x\in[0,1],\, T(f)(x):=\int^x_0 f(t) dt.$$
%T ist offenbar linear und beschränkt. T ist injektiv, denn es gilt:
%$$  T(f)(x)=\int^x_0 f(t) dt = 0 \;\forall x\in [0,1] \Leftrightarrow f = 0$$
%\glqq $\df$\grqq:
%Angenommen $f \not = 0$, so ist $f$ wegen der Stetigkeit in einer Umgebung $\not = 0$, zum Beispiel $>0$, dann wäre aber das Integral für $x$ aus dieser Umgebung $\not = 0$ ($f$ kann auch nicht konstant sein).\\
%\glqq $\Leftarrow$\grqq: Dies folgt direkt.\par
%Außerdem ist $\im T \subset C^0_0[0,1]$. Dies folgt aus dem Hauptsatz der Integral- und Differentialrechnung. Insbesondere ist $Tf$ stetig differenzierbar. Allerdings gilt nicht \glqq =\grqq, wir betrachten dafür die nicht stetig differenzierbare Abbildung $xsin(\frac{1}{x})$. Es ist $\im T = C^1_0[0,1]$, wir können ja jede Funktion aus $C^1_0[0,1]$ einmal ableiten und dann integrieren.
	

\item $T_2 \in \B(X)$ injektiv mit $\overline{\im T} \not =X$.\par 
	1. Variante:
	$X := \ell^1$, $T_2(x_1,x_2,x_3,\dots) := (0,x_1,x_2,x_3,\dots)$. $T$ erfüllt offenbar alle Bedingungen.\par\bigskip
	2.Variante:
	Wie in a) 2.Variante aber wir erweitern $X$ auf die Menge aller Riemann-Integrierbaren Funktionen mit $f(0)=0$, und bilden geschickt Äquivalenzklassen wie im $L^p$. Dadurch wird die Injektivität gewährleistet, und das Bild ist dank dem Hauptsatz der Integral- und Differentialrechnung eine Teilmenge der stetigen Funktionen, welches mit der Supremumsnorm nicht dicht in $X$ liegen kann.

%Erstmal können wir für $\mathcal{R}[0,1]$ genau so Aquivalenzklassen bilden wie in $L^p$ und zwar mit:
%$$f\sim g :\Leftrightarrow \int^x_0 (f-g)(x) dx =0$$
%Es sei $L^\infty_0[0,1] :=\{f\in L^\infty[0,1]:f(0)=0\}.$
%Wir betrachten $(X,\|\cdot\|)=(\mathcal{R}_0[0,1],\|\cdot\|_\infty)$ die Menge der Riemann-integrierbaren Funktionen mit $f(0)=0$. Und
%$$T:X\to X,\;\text{ mit } f\in X,\, x\in[0,1],\, T(f)(x):=\int^x_0 f(t) dt.$$

\item $T_3 \in \B(X)$ surjektiv, aber nicht injektiv.\par 
	$X := \ell^1$, $T_3(x_1,x_2,x_3,\dots) := (x_2,x_3,x_4\dots)$. Wegen $T_3\circ T_2 = id$ ist $T_3$ surjektiv. Alle anderen Bedingungen sind natürlich auch erfüllt.

\end{enumerate}

%%%%%%%%%%%%%%%%% Aufgabe 2
\subsection{$\ell^p$ und seine \afs Basis \afs}
Wir definieren den k-ten kanonischen Einheitsvektor $e_k\in\ell^p$ durch $e_k:= (\delta_{nk})_{n\in\N}$
\begin{beh}
Für $p\in[1,\infty)$ gilt $$\overline{\aufspan (e_k)_{k\in\N}} = \ell^p.$$
Für $p= \infty$ hingegen gilt
	$$\overline{\aufspan (e_k)_{k\in\N}} = c_0.$$
\end{beh}
\begin{proof}
Wir erinnern uns vorerst an die Definition von $\aufspan$: 
$$\aufspan (e_k)_{k\in\N} = \left\lbrace\sum_{k\in M} a_k e_k : a_k \in \K,\,k\in M,M \subseteq	 \N ,\, |M| < \infty \right\rbrace$$
$\aufspan (e_k)_{k\in\N}$ ist also die Menge aller \textit{endlichen} Linearkombinationen von $(e_k)_{k\in\N}$.
Nun zum eigentlichen Beweis:\par
Erstmal für $p\in [1,\infty)$\\
\glqq $\subseteq$ \grqq:\\
$\aufspan (e_k)_{k\in\N} \subseteq \ell^p$ und $\ell^p$ ist abgeschlossen. \par

\glqq $\supseteq$ \grqq:\\
Sei $(x_k) \in \ell^p$ d.h $$\sum^\infty_{k=1} |x_k|^p < \infty \Leftrightarrow 
\sum^\infty_{k=n} |x_k|^p \longrightarrow 0 \text{ für } n\longrightarrow\infty \text{ (Cauchy-Kriterium)}$$

Wir definieren $(y^n_k)\in \aufspan (e_k)_{k\in\N}$ durch $\displaystyle y^n_k := x_k e_k$ für $k=1,\dots,n$ und $y^n_k = 0$ für $k>n$.
$$\df \|(x_k)-(y^n_k)\|^p_p = \sum^\infty_{k=1}|x_k-y^n_k|^p = \sum^\infty_{k=n} |x_k|^p\longrightarrow 0 \text{ für } n\longrightarrow \infty$$
Das bedeutet: $(y^n_k) \longrightarrow (x_k)$. Da $\overline{\aufspan (e_k)_{k\in\N}}$ abgeschlossen ist, ist $(x_k) \in \overline{\aufspan (e_k)_{k\in\N}}$.

$p = \infty$\\
\afs $\subseteq$ \afs:\\
Wir zeigen die Abgeschlossenheit von $c_0$. Dafür sei $((x^{(n)}_k)_{k\in\N})_{n\in\N}$ eine Folge (aus Folgen) aus $c_0$, mit $\limes (x^{(n)}_k) = c$. Zu zeigen ist $c\in c_0$. Sei $\varepsilon>0$ beliebig
$$\df \norm{x^{(n)}_k - c}_\infty < \frac{1}{2}\varepsilon,\; \forall n \geq N\left(\frac{1}{2}\varepsilon\right)\in\N 
\df \forall k\in\N : \abs{x^{(n)}_k -c_k}<\frac{1}{2}\varepsilon,$$
	$$\lim_{k\to\infty} x^{(n)}_k = 0 \df \exists M\in\N : \forall k \geq M\; \abs{x^{(n)}_k} < \frac{1}{2}\varepsilon$$
	$$\df \abs{c_k} = \abs{c_k - x^{(n)}_k + x^{(n)}_k} \leq \abs{x^{(n)}_k -c_k} + \abs{x^{(n)}_k} < \varepsilon \df \lim_{k\to\infty} c_k = 0 \df c \in c_0.$$ 
\afs $\supseteq$ \afs:\\
	Analog zu $p <\infty$.
\end{proof}

\subsection{Offene Abbildung und seine Äquivalenzen}
Es seien $(X,\norm{\cdot}_X),(Y,\norm{\cdot}_Y)$ zwei normierte Vektorräume und $T:X\to Y$ linear.
\begin{beh}
Es ist äquivalent:
\begin{enumerate}[(a)]
	\item $T(U) \subseteq Y$ ist offen für alle offenen $U\subseteq X$.
	\item Für alle $r>0$ existiert ein $\varepsilon>0$, so dass $V_\varepsilon(0) \subseteq T(U_r(0)).$
	\item Es gibt ein $\varepsilon > 0$, so dass $V_\varepsilon(0) \subseteq T(U_1(0))$.
\end{enumerate}
Falls $T$ bijektiv ist, dann sind die obigen Aussagen äquivalent dazu, dass die Inverse von $T$ beschränkt ist.
\end{beh}

\begin{proof}
	(a) $\df$ (b) $\df$ (c) trivial \textbf{(Muahahaha)}.\par 
 	\afs (c) $\df$ (a)\afs : Es sei $U\subseteq X$ offen, und $y \in T(U)$. Es ist zu zeigen, dass es eine Umgebung um $y$ gibt, die in $T(U)$ enthalten ist. Es gibt ein $x \in U$ mit $Tx = y$. Da $U$ offen ist, ist auch $U - x$ offen und eine Nullumgebung. Es gibt also ein $\varepsilon>0$, so dass $U_\varepsilon := U_\varepsilon(0) \subseteq U - x \df T(U_\varepsilon) = T(\varepsilon U_1) = \varepsilon T(U_1) \supseteq \varepsilon V_\delta (0)$ für ein geeignetes $\delta > 0$. Nun ist 
 	$$\varepsilon V_\delta(0) \subseteq T(U_\varepsilon) \subseteq T(U-x) = T(U) - y \Leftrightarrow T(U) \supseteq y + \varepsilon V_\delta(0)  = \varepsilon V_\delta(y)$$
 	also haben wir eine Umgebung um $y$ gefunden.\par
	Sei nun $T$ bijektiv, und $U \subseteq X$ offen: 
	$$ T \text{ offen } \Leftrightarrow T(U) = {T^{-1}}^{-1}(U) \text{ offen } \Leftrightarrow T^{-1}:Y\to X \text{ stetig} \Leftrightarrow T^{-1} \text{ beschränkt.}$$	
\end{proof}

%%%%%%%%%%%%%%%%%%%%%%%%%%%%%%%%%%%%%%%%%%%%
%%% Blatt 3
%%%%%%%%%%%%%%%%%%%%%%%%%%%%%%%%%%%%%%%%%%%%

\newpage
\section{Blatt 3}
\subsection{Resolventenmenge, Spektrum}
Es seien $(X,\norm{\cdot }_X),\, (Y,\norm{\cdot }_Y)$ Banachräume über den Skalarkörper $\K$. 
\begin{beh}
Es gelten die Aussagen
	\begin{enumerate}[(a)]
	\item Die Menge $M:=\{T\in \B(X,Y) : T \text{ bijektiv mit } T^{-1} \in \B(Y,X)\}$ ist offen in $B(X,Y)$.
	
	\item Für $T\in \B(X)$ ist die \textit{Resolventenmenge} 
	$$\rho(T):=\{ \lambda \in \K : \lambda I - T \text{ bijektiv mit } (\lambda I - T)^{-1}  \in \B(X)\}$$ 
	offen in $\K$.
	
	\item Für $n\in\N$ ist $\rho(T^n) = \rho(T)^n:=\{ \lambda^n : \lambda \in \rho(T)\} $
	
	\item Für $T \in \B(X)$ gilt, dass $\{ \lambda \in \K : \abs{\lambda} > \| T\| \} \subseteq \rho(T)$
		
	\item $\sigma(T) := \rho(T)^c = \K\setminus \rho(T)$ heißt \textit{Spektrum} von $T$. Der \textit{Spektralradius}
	$r(T):= \sup \{ \abs{\lambda} : \lambda \in \sigma(T) \}$ ist endlich und $r(T) = \max \{ \abs{\lambda} : \lambda \in \sigma(T) \} $.
	
	\item Es ist $r(T) \leq \inf_{n\in \N} \norm{T^n}^{\frac{1}{n}}$
	\end{enumerate}
\end{beh}

\begin{proof}
	\begin{enumerate}[(a)]
	\item Es sei $T \in M$, $R:= \frac{1}{\norm{T^{-1}}}$ und $ S\in U_R(T)\subseteq \B(X,Y)$ 
	$$\df \norm{ T^{-1} (T-S)} \leq \norm{T^{-1}} \norm{T-S} < 1.$$
	Nach Satz 1.25 (Neumannsche Reihe) ist $I-T^{-1}(T-S)$ invertierbar mit stetiger Inverse. Nun ist 
	$S = T(I-T^{-1}(T-S))$ bijektiv als Verkettung bijektiver Funktionen und hat eine stetige Inverse, da auch die einzelnen Funktionen eine stetige Inverse haben $\df S\in M$.
	
	\item Es sei $\lambda \in \rho(T)$. Nach (a) gibt es ein $r>0$, so dass für alle $S\in U_r(T),\,\lambda I -T+S$ bijektiv ist und eine stetige Inverse hat. Für $\mu \in \R,\abs{\mu} < r$ ist $\norm{\mu I} = \abs{\mu} < r$. Da $\lambda I + \mu I -T = \lambda I - T + \mu I$ bijektiv ist und eine stetige Inverse hat, ist $\lambda + \mu \in \rho(T) \df (\lambda - r,\, \lambda +r) \subseteq \rho(T)$. 
	
	\item TODO
	
	\item Es sei $\lambda \in \K,\, |\lambda| > \|T\| \df \|\frac{1}{\lambda} T \| = \frac{1}{|\lambda|} \| T \| < 1$. Damit ist $\lambda I - T = \lambda ( I - \frac{1}{\lambda}T)$ nach Satz 1.25 invertierbar und somit $\lambda \in \rho(T)$. Hieraus folgt insbesondere, dass die Resolventenmenge unbeschränkt ist.
	%$ \, R:= \frac{1}{\norm{(\lambda I - T )^{-1}}$ und $\mu \in U_R(\lambda)$
	
	\item $\sigma(T)$ lässt sich wegen (d) nach oben durch $\|T\|$ beschränken. Damit ist auch $r(T)$ endlich. Da $\rho(T)$ nach (b) offen ist, ist $\sigma(T)$ abgeschlossen. Damit wird auch das Maximum angenommen.
	
	\item TODO
	\end{enumerate}
\end{proof}

\subsection{Fredholmoperator}
	Es seien $a<b$ reelle Zahlen, $k\in C([a,b]^2)$ und der \textit{Fredholmoperator} $K:C([a,b]) \to C([a,b])$ gegeben durch 	
	$$ (Kf)(s) = \int^b_a k(s,t)f(t) dt.$$
	\begin{beh}
		Es gelten die folgenden Aussagen:
		\begin{enumerate}[(a)]
			\item $K\in \B([a,b])$ und $\| K \| \leq \max_{s\in[a,b]}\|k(s,\cdot) \|_{L^1(a,b)}$
			\item Es existiert $c_0 > 0$, so dass für $|\lambda| > c_0$ die \textit{Fredholm'sche Integralgleichung}
			$$(\lambda I - K)f = g$$
			für jedes $g \in C([a,b])$ eine eindeutige Lösung $f\in C([a,b])$ hat. Weiterhin gilt, dass die Abbildung $f\mapsto g$ stetig ist.
		\end{enumerate}
	\end{beh}
	\begin{proof}
		\begin{enumerate}[(a)]
		\item Es sei $f \in C([a,b]),\, \|f\|_{\infty} < 1$.
		\begin{equation*}
		\begin{split}
		\df \abs{\int^b_a k(s,t)f(t) dt} 
		& \leq \int^b_a \abs{k(s,t)f(t)} dt 
		\\ & \leq \int^b_a \max_{s\in [a,b]} |k(s,t)| \max_{x\in[a,b]} |f(x)| dt 
		\\ & \leq \int^b_a \max_{s\in [a,b]} |k(s,t)| dt
		\\ & \overset{?}{\leq} \max_{s\in [a,b]} \int^b_a |k(s,t)|dt = \max_{s\in[a,b]} \| k(s,\cdot) \|_{L^1(a,b)}
		\end{split}
		\end{equation*}
		%$$\df \abs{\int^b_a k(s,t)f(t) dt} \leq \int^b_a \abs{k(s,t)f(t)} dt \leq \int^b_a \max_{s\in [a,b]} |k(s,t)| \max_{x\in[a,b]} |f(x)| dt \leq \int^b_a \max_{s\in [a,b]} |k(s,t)| \overset{?}{\leq} \max_{s\in [a,b]} \int^b_a |k(s,t)|dt$$
		Wenn wir $\overset{?}{\leq}$ zeigen können, ist alles gezeigt. Aber das verschieben wir auf später \dots
		
		\item Wegen (a) ist $K$ beschränkt. Wählen wir $c_0 := \|K\|$ so ist wegen Aufgabe 1 (d) $(\lambda I - K)$ bijektiv. Damit gibt es insbesondere stets eine eindeutige Lösung. Die Abbildung ist linear, wegen der Beschränktheit also auch stetig. 		
		
		\end{enumerate}
	\end{proof}

%%%%%%%%%%%%%%%%%%%%%%Aufgabe 3
\subsection{Der \afs Links-Shift\afs}
	Es sei 
	$$S_l : \ell^p \to \ell^p,\quad (x_1,x_2,\dots)\mapsto S_l(x_2,x_3,\dots)$$
	der \textit{Links-Shift}.
	\begin{beh}
		Es ist: 
		$$\|S_l\| = 1,\quad \rho(S_l) = \{z\in\C : |z| > 1\} ,\quad r(S_l) = 1$$
	\end{beh}
	\begin{proof}
		Für $x\in \ell^p $ ist
		$$\norm{S_l(x)}_p = \sqrt[p]{\sum^\infty_{n=2} \abs{x_n}^p} \leq \sqrt[p]{\sum^\infty_{n=1} \abs{x_n}^p } = \norm{x}_p \df \norm{S_l} \leq 1$$
		und für $e_2= (0,1,0,\dots) \df \|S_l(e_2)\|_p = \| e_1 \|_p = 1 \df \|S_l\| \geq 1$. 	Zusammen ist also $\|S_l\| = 1$.\\
		Wegen Aufgabe 1 d) ist schon mal $\rho(S_l) \supseteq \{z\in\C:\abs{z}>1\}$. Für $\lambda\in \C,\,|\lambda|<1$ zeigen wir, dass $\lambda I - S_l$ nicht injektiv sein kann. Es ist nämlich:
		$$(\lambda I - S_l)x = 0 \Leftrightarrow \lambda x = S_lx \Leftrightarrow (\lambda x_1, \lambda x_2, \dots ) = ( x_2, x_3  ,\dots) \Leftrightarrow x = x_1(1,\lambda,\lambda^2,\dots) \in \ell^p  \Leftrightarrow |\lambda| < 1 $$
		Das heißt kern$(\lambda I - S_l) \not = \{0\}$ also nicht injektiv. Da $\rho(S_l)$ wegen Aufgabe 1 b) offen ist, ist $\rho(S_l) =  \{z\in\C : |z| > 1\}$. Daraus folgt auch direkt:
		$$r(S_l) = \max \{ \abs{\lambda} : \lambda \in \rho(T)^c \} = \max \{ \abs{\lambda} : |\lambda| \leq 1 \} = 1$$
	\end{proof}
	
%%%%%%%%%%%%%%%%%%%%%%%%%%%%%%%%
%%%%% Blatt 4
%%%%%%%%%%%%%%%%%%%%%%%%%%%%%%%%

\newpage
\section{Blatt 4}
%%%%%%%%%%%%% Aufgabe 1
\subsection{Norm auf dem Quotientenraum}
Es sei $(X, \|\cdot \|)$ ein normierter Raum und $E\subseteq X$ ein Unterraum.
\begin{beh}
	\begin{enumerate}[(a)]
	\item 	Die Abbildung
	$$ \semi{\cdot} : X/E \to \R,\quad x+E \mapsto \inf_{e\in E} \norm{x+e}$$
	definiert eine Halbnorm auf $X/E$.
	
	\item 	$\semi{\cdot}$ definiert eine Norm auf $X/E$, wenn $E$ abgeschlossen ist. 	
	\marginpar{(Äquivalentes Kriterium ?!)} 
	
	\item $(X/E, \semi{\cdot})$ ist ein Banachraum, wenn $E$ vollständig ist.
	
	\end{enumerate}
\end{beh}
\begin{proof} 
	\begin{enumerate}[(a)]
	\item 
		Wir müssen beweisen: 
	$$(i)\;\semi{0} = 0,\quad(ii)\;\semi{\alpha x} = |\alpha| \semi{x},\quad (iii)\;\semi{x+y}\leq \semi{x}+\semi{y}$$
	für $\alpha \in \K,\; x,y\in X/E$.
	\begin{equation*}
		\begin{split}
		& (i):\quad \semi{0} = \inf_{e\in E}\|0+e\| \leq \|0\| = 0
		\\ & (ii):\quad  \alpha = 0\df (i),\;\alpha \not = 0: \semi{\alpha x} = \inf_{e\in E}\|\alpha x+e\| = \abs{\alpha} \inf_{e\in E}\| x+\frac{1}{\alpha} e\| \overset{\frac{1}{\alpha}e\in E}{=} |\alpha| \semi{x}
		\\ & (iii):\quad \semi{x+y} = \inf_{e\in E}\|x+y+e\| \overset{e_1+e_2=e}{=} \inf_{e_1,e_2\in E}\|x+y+e_1+e_2\| \leq \\ & \quad\quad\quad \inf_{e\in E}\|x+e\| + \inf_{e\in E}\|y+e\| = \semi{x}+ \semi{y}
		\end{split}
	\end{equation*}
	
	\item 
		Wir müssen nur noch $\semi{x+E} = 0 \df x + E = 0$ zeigen. Dafür erinnern wir uns erstmal, was $0 \in X/E$ bedeutet. Aus der Linearen Algebra ist bekannt, dass $x+E=y+E \Leftrightarrow x - y \in E$. Es sind also genau die Elemente aus $E$ die Nullelemente, bzw. $E$ ist das Nullelement.\\
	2.Variante\\
	Sei nun $\semi{x+E} = 0$. Es gibt also eine Folge $(x_n)$ aus $E$, mit $x_n \to -x$ für $n\to \infty$, da nur so das Infinum $0$ annehmen kann. Da $E$ abgeschlossen ist, ist $-x\in E$ also (nach oben) $ x = 0$.\par 
	3.Variante\\
	Wie 2. Variante, aber wir wählen eine Folge $(x_n)$ aus $x+E$ mit $x_n \to 0$. Da $E$ abgeschlossen ist, ist auch $x+E$ abgeschlossen, und $0\in x+E$.
	
	\item 
	$X/E$ ist wegen der Linearen Algebra schon ein Vektorraum. Wegen b) ist $X/E$ normiert. Die Vollständigkeit ist trivial.
	
	\end{enumerate}
\end{proof}




%%%%%%%%%%%%%%%%%%%% Aufgabe 2
\subsection{Vektorräume mit abzählbaren Basen}
Es sei $(X,\norm{\cdot})$ ein normierter Vektorraum mit abzählbar unendlicher Basis $B:=\{b_i : i\in\N\}$.
\begin{beh}
	$X$ ist nicht vollständig.
\end{beh}
\begin{proof}
	Wir nehmen O.B.d.A an, dass $\norm{b_i} = 1,\;\forall i\in\N$
	und definieren $U_n := \aufspan \{b_1,\dots,b_n\}$. Es gilt für alle $n\in\N$:
	\begin{enumerate}
		\item $U_n$ ist abgeschlossen
		\item $\mathring{U_n}=\emptyset$
		\item $\bigcup_{n\in\N} U_n = X$
	\end{enumerate}
	\begin{enumerate}[1:]
		\item $U_n$ ist endlichdimensional also abgeschlossen.
		\item Per Widerspruch: Angenommen, es gäbe ein $x\in U_n$ und ein $\varepsilon>0$ so dass $U_\varepsilon(x)\subseteq U_n$. Dann ist aber 
		$$v := x - \frac{\varepsilon}{2} b_{n+1}\not \in U_n$$
		 wegen der linearen Unabhängigkeit der $b_i$, außerdem ist 
		 $$\norm{x-v} = \norm{x-x+\frac{\varepsilon}{2}b_{n+1}} = \norm{\frac{\varepsilon}{2} b_{n+1}} < \varepsilon$$
		 Also $v\in U_\varepsilon(x)\df U_\varepsilon(x) \not \subseteq U_n$.
		\item Offenbar gilt $\afs\subseteq\afs$. $\afs \supseteq \afs$: Für $x\in X$, ist $x=\sum^k_{i=1} \alpha_ib_i,\;\alpha_i\in\K$, also $x\in \bigcup^k_{n=1} U_n$
	\end{enumerate}
	Nun folgt mit dem Baireschen Kategoriensatz 1.35 die Nicht-Vollständigkeit von $X$. Wäre nämlich $X$ vollständig, müsste wegen 1. und 3. der Satz gelten, 
	und ein $U_n$ hätte kein leeres Inneres, was im Widerspruch zu 2. steht $\lightning$
\end{proof}

%%%%%%%%%%%%%%%%%%%%%%% Aufgabe 3
\subsection{Kompaktheit in endlichdimensionalen Vektorräumen}
$(V,\norm{\cdot})$ sei ein endlichdimensionaler Vektorraum über den Körper $\K$ und $K\subseteq V$.
Und es gelten die Resultate:
	\begin{enumerate}[(1.)]
	\item Je zwei Normen auf $\K^n$ sind äquivalent.
	
	\item Eine Teilmenge von $\K^n$ versehen mit der Euklidischen Norm ist genau dann kompakt, wenn sie beschränkt und abgeschlossen ist.	
	\end{enumerate}
\begin{beh}
	$$K \text{ kompakt} \Leftrightarrow K \text{ abgeschlossen und beschränkt}$$
\end{beh}
\begin{proof}
	$\afs\df\afs$: Kompakte Mengen sind beschränkt und abgeschlossen.\par
	$\afs \Leftarrow \afs$: 
	Da auf dem $\K^n$ je zwei Normen äquivalent sind (1.) und bzgl. einer Normänderung die topologischen Eigeschaften Kompaktheit, Abgeschlossenheit und Beschränktheit nicht geändert werden
	\footnote{Die Übertragung der Abgeschlossenheit und Beschränktheit ergibt sich direkt aus der Definition und Satz 1.18. Die Kompaktheit ergibt sich auch, denn eine beschränkte Folge hat bzgl. der Euklidischen Norm stets einen partiellen Grenzwert, dank der Übetragung der Konvergenz bleibt der partielle Grenzwert erhalten.}	
	, gilt (2.) auch für alle anderen Normen. Sei $n:= \dim V$, wegen der Linearen Algebra gibt es einen Isomorphismus $\varphi:V\to \K^n$, insbesondere ist $\varphi^{-1}$ stetig. Da $\varphi(K)\subseteq \K^n$ abgeschlossen und beschränkt ist, ist es kompakt. Also auch $\varphi^{-1}(\varphi(K))=K$.	
\end{proof}

%%%%%%%%%%%%%%%%%%%%%%%%%%%%%%%%%%%%%%%%%%
%%%%%%%%%% Blatt 5
%%%%%%%%%%%%%%%%%%%%%%%%%%%%%%%%%%%%%%%%%%
\newpage
\section{Blatt 5}

\subsection{ Beispiel für einen Hilbertraum}
Es sei $J$ eine beliebige nichtleere Menge und
	$$X:=\{ f: J \to \R : f(j) \not = 0 \text{ für höchstens abzählbar viele } j\in J,\; \sum_{j\in J} \abs{f(j)} ^2 < \infty\}.$$
	\begin{enumerate}[(a)]
	\item Sei $f,g \in X$
		$$(f,g) := \sum_{j\in J} f(j)g(j) $$
		ist ein Skalarprodukt auf $X$, mit dem $X$ zu einem Hilbertraum wird.
	
	\item Die Familie $(e_j)_{j\in J}$ mit $e_j(k) := \delta_{jk} $ für $j,k\in J$ bildet eine Orthonormalbasis von $X$.
	
	\item $X$ ist separabel $\Leftrightarrow$ $J$ ist höchstens abzählbar.	
	\end{enumerate}
	
	\begin{proof}
	\begin{enumerate}[(a)]
		\item tüddellich ... TODO
		\item TODO
		\item TODO
	\end{enumerate}		
	
	\end{proof}















\end{document}


























