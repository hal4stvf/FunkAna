	\chapter{Funktionenräume}
	\section{Dualität in L$^p$}

	$p\in [1,\infty]$, $(\Omega, \mathcal{A}, \mu)$ Maßraum. $q\in [1,\infty]$, so dass $\frac{1}{p} + \frac{1}{q} = 1$. Nach Hölder-Ungleichng gitl für $f\in L^p(\Omega,\mu), g\in L^q(\Omega, \mu)$.\\
	$$\left| \int_\Omega fg d\mu \right| \leq \norm{f}_p \norm{g}_q$$
	Mit anderen Worten. Die Abbidung
		$$ J_p : L^q(\Omega,\mu) \to L^p(\Omega, \mu)' $$
		$$ g \mapsto (f \mapsto \int_\Omega fg d\mu)$$
	definiere lineare Abbildung mit $\norm{J_pg} \leq \norm{g}_q$.\\
	Für $p=2$ gilt : $J_p = \ov{R_{L^2}}$, insbesondere ist $J_p$ normerhaltend und surjektiv. (S.v. Frechet-Riesz). Sonst?

	\begin{thm}
		$p\in [1,\infty)$, $q\in [1,\infty]$, so dass $\frac{1}{p} + \frac{1}{q} = 1$. 
		Dann ist die Abbildung 
			$$J_p : \ell^q \to (\ell^p)'$$
			$$ (y_n) \mapsto ((x_n) \mapsto \sum_{n=1}^\infty x_ny_n)$$
		normerhaltend und surjektiv.
	\end{thm}

	Gilt auch $imJ_\infty = (\ell^\infty)'$?

	\begin{cor}
		Die Abbildung 
		$$J_\infty : \ell^1 \to (\ell^\infty)'_\infty$$
		$$(y_n) \mapsto ((x_n) \mapsto \sum_{n=1}^\infty x_ny_n)$$
		ist normerhaltend, jedoch nicht surjektiv.
	\end{cor}

	\begin{definition}
		$\Omega$ Menge, $\mathcal{R} \subseteq \mathcal{P}(\Omega)$. 
		Ring über $\Omega$ (d.h. 
		$$\emptyset \in \mathcal{R}, A,B \in \mathcal{R} \df A\backslash B, A\cup B \in \mathcal{R}).$$
		 Eine Abbildung : 
			$$ \mu: \mathcal{R} \to \K \quad \text{ heißt}$$ 
		\begin{enumerate}[a)]
			\item "'additiv"', wenn $\mu(\emptyset) = 0, \mu(A\cup B) = \mu(A) + \mu(B) \forall A,B \in \mathcal{R}$ mit $A\cap B = \emptyset$.
			\item "'$\sigma$-additiv"'
			\item "'signiertes Maß "', wenn $\mathcal{R}$ $\sigma$-Algebra ist $\K = \R$ und $\mu$ $\sigma$-additiv
			\item "'komplexes Maß"'
			\item "'Maß"'
			\item "'$\sigma$-endliches Maß"'
			\item "'endliches Maß "'
		\end{enumerate}
	\end{definition}

	\begin{thm}
		$(\Omega, \mathcal{A}, \mu)$ Maßraum. $p\in [1, \infty]$, $q\in [1,\infty]$, so dass $\frac{1}{p} + \frac{1}{q} = 1$. Betrachte die Abbildung
		\begin{equation}
		\begin{split}
			 J_p : & L^q(\Omega,\mu) \to L^p (\Omega, \mu)',\\ &
			 g \mapsto (f \mapsto \int_\Omega fg d\mu) 		
		\end{split}
		\end{equation}
		Dann gilt 
			\begin{enumerate}
				\item $J_p$ normerhaltend
				\item Falls $p \in (1,\infty)$, so ist $J_p$ surjektiv
				\item Falls $p = 1$ und $\mu$ $\sigma$-endlich, dann ist $J_p$ surjektiv.
			\end{enumerate}
	\end{thm}
	\begin{proof}
		Ohne Beweis. Siehe Alt.
	\end{proof}
	
	\begin{cor}
		$(\Omega, \mathcal{A}, \mu)$ Maßraum, $p\in (1,\infty)$. Dann ist $L^p (\Omega, \mu)$ reflexiv.
	\end{cor}
	\begin{proof}
		Wir betrachten die Abbildungen 
		$$J_p : L^q(\Omega,\mu) \to L^p(\Omega,\mu)',\qquad J_q:L^p(\Omega,\mu)\to L^q(\Omega,\mu)'$$	
		gemäß (4.1) . Dann gilt:
		\todo{richtige Querverweise?}
		\begin{equation}
			\fop{f,J_p g} = \int_\Omega fgd\mu  = \fop{g,J_q f}
		\end{equation}
		Sei $f'' \in L^p(\Omega,\mu)''$, da $L^q(\Omega,\mu) \to \K,\quad g \mapsto \fop{J_pg, f''}$ ein linearer Funktional aus $L^q(\Omega,\mu)'$ beschreibt, können wir auch schreiben:
		\begin{equation}
			\fop{g,f'} = \fop{J_p g, f''},\quad \forall g\in L^q(\Omega, \mu)		.	
		\end{equation}
		Sei $f:= J^{-1}_q f' \in L^p(\Omega,\mu).$ Dann gilt für alle $g'\in L^p(\Omega,\mu)'$
		$$\fop{g',f''} = \fop{J_p { J^{-1}_p g'} , f''} \overset{(4.3)}{ = } \fop{J^{-1}_p g',f'} = \int_\Omega (J^{-1}_p g') f d \mu = \int_\Omega f (J_p^{-1} g' \overset{(4.2)}{=} \fop{f,g'}.$$
	\end{proof}		

	\begin{cor}
		Sei $\Omega \subseteq \R^n$ offen und nichtleer. Dann ist
			$$J_\infty : L^1(\Omega) \to L^\infty(\Omega)'$$
			nicht surjektiv.
	\end{cor}
	\begin{proof}
		Sei $BC:= C(\Omega,\K) \cap L^\infty(\Omega) = \{f:\Omega \to \K : f$ stetig und beschränkt $\}$ und für
		$x_0 \in \Omega$ definieren wir
		$$f: BC(\Omega) \to \K,\quad \varphi \mapsto \varphi(x_0)$$
		$f$ ist offenbar linear und  wegen $|\fop{\varphi, f}| = |\varphi (x_0)| \leq \norm{\varphi}_\infty$
		beschränkt, also $\in L^\infty(\Omega)'$. Da $\fop{1_\Omega,f} = 1$, ist $\norm{f} = 1$. Nach Hahn-Banach gibt es also ein $x' \in L^\infty(\Omega)'$, so dass $\|x'\| = 1$ und $x'|_{BC(\Omega)} = f$.\\
		Wir zeigen nun $x' \not\in im J_\infty$ per Widerspruch: Angenommen $x' \in im J_\infty \df \exists g \in L^1(\Omega) : J_\infty g = x'$
		$$\df \forall \varphi\in BC(\Omega) : \varphi(x_0) = \fop{\varphi, x'} = \int_\Omega \varphi g d\lambda^n$$
		Sei $(\varphi_k)$ eine Folge in $BC(\Omega)$ mit 
		$$\supp\, \varphi_k \subseteq U_{\frac{1}{k}}(x_0),\quad \norm{\varphi_k}_\infty = 1,\quad \varphi_k(x_0) = 1\quad \forall k\in\N$$
		Dann ist aber
		$$1 = |\fop{\varphi_k, x'}| = \abs{\int_\Omega \varphi_k g d\lambda^n}\leq \int_\Omega |\varphi_k||g| d\lambda^n \leq \| \varphi_k \|_\infty \int_{U_{\frac{1}{k}}(x_0)}|g|d \lambda^n = \int_{U_{\frac{1}{k}}(x_0)}|g|d \lambda^n \overset{k\to\infty}{\longrightarrow} 0 \;\lightning!$$
		
		Wobei sich die Konvergenz gegen $0$ durch $\int_{U_{\frac{1}{k}}(x_0)}|g|d \lambda^n = \int_\Omega 1_{U_{\frac{1}{k}}(x_0)} |g| d\lambda^n$ und dem Satz der monotonen Konvergenz erklärt. 
	\end{proof}



%%%%%%%%%%%%%%%%%%%%%%%%%%%%%%%%%%%%%%%%%%%%%%%%%%%%%%%%%%%%%%%%%%%%%%%%%%%%%%
	\section{Der Dualraum von $C(K,\K)$}
%%%%%%%%%%%%%%%%%%%%%%%%%%%%%%%%%%%%%%%%%%%%%%%%%%%%%%%%%%%%%%%%%%%%%%%%%%%%%%

	In diesem Abschnitt wollen wir den Dualraum von $C(K)$ für eine kompakte, hausdorffsche Menge $K$ mit der Topologie $\tT \in P(K)$ näher untersuchen. Dafür benötigen wir einiges an Maßtheorie.
	Wir wissen bereits, dass $C(K):=C(K,\K)$ mit der Supremumsnorm zu einem Banachraum wird. $\B = \sigma(\tT)$ ist die \textit{Borel $\sigma$-Algebra}, also die kleinste $\sigma$-Algebra, die noch $\tT$ enthält. Entsprechend ist $\B_0$ der kleinste Ring, der $\tT$ enthält.
	\begin{definition}[inkl. Satz]
		Sei $R$ ein Ring über $\Omega$ mit $\Omega\in R$ (also eine Algebra), $\mu R\to\K$ additiv. Für $E\in R$ definieren wir
		$$|\mu |(E) := \sup \left\lbrace \sum^k_{i=1} |\mu(E_i)|: k\in\N,\, E_1,\dots,E_k\in R \text{ paarweise disjunkt, } E_i \subseteq E \right\rbrace$$
		das \textit{Variationsmaß zu $\mu$}. Das Variationsmaß $|\mu| : R \to [0,\infty]$
		ist additiv.
		$$\| \mu \|_{Var} := |\mu|(\Omega)$$
		heißt \textit{Totalvariation  zu $\mu$}. $\mu$ heißt \textit{beschränkt}, falls $\| \mu \|_{Var} <\infty$.
	\end{definition}
	\begin{proof}[der Additivität von $|\mu|$]
	Seien $B_1, B_2 \in R$ disjunkt.
	Es ist zu zeigen:
		$$\df |\mu| (B_1)+|\mu| (B_2) = |\mu|(B_1\cup B_2)$$
		Die Ungleichung \afs $\leq$ \afs ist klar.
		Für die andere Ungleichheit sei $\varepsilon>0$ beliebig. Seien $E_1,\dots,E_k \in R$ paarweise disjunkte Mengen mit $E_i \subseteq B_1 \cup B_2$, so dass
		$$
		|\mu| (B_1 \cup B_2) - \varepsilon \leq \sum^k_{i=1} |\mu|(E_i)|
		$$
		$$\df \sum^k_{i=1} |\mu|(E_i)| \leq  \sum^k_{i=1} |\mu|(E_i \cap B_1)+ \mu(E_i \cap B_2)| \leq \sum^k_{i=1} |\mu|(E_i \cap B_1)| + \sum^k_{i=1} |\mu(E_i \cap B_2)| \leq |\mu|(B_1) + |\mu|(B_2)$$
		Insgesamt also $|\mu| (B_1 \cup B_2) - \varepsilon \leq |\mu|(B_1) + |\mu|(B_2)$. Da $\varepsilon > 0$ beliebig war, folgt die Behauptung.		
	\end{proof}
	
	\begin{definition}
	$K$ sei ein kompakter Hausdorffraum mit der Topologie $\tT$. Wir definieren
	\begin{equation*}
		\begin{split}
		& ba(K) := \{\mu : B_0 \to \K : \mu \text{ additiv und beschränkt} \}\\
		& ca(K) := \{\mu : B \to \K : \mu\; \sigma-\text{additiv und beschränkt} \}
		\end{split}
	\end{equation*}
		$M\in ba(K)$ heißt \textit{regulär}, wenn für alle $E\in \tT$ gilt
		$$\inf \left\lbrace |\mu| (U\setminus C) : C\subseteq E \subseteq U,\, C \text{ abgeschlossen, } U \text{ offen} \right\rbrace = 0.$$
		Zusätzlich definieren wir
		$$rba := \left\lbrace \mu \in ba(K) : \mu \text{ regulär}\right\rbrace,\qquad 
		rca := \left\lbrace \mu \in ca(K) : \mu \text{ regulär}\right\rbrace. $$
		Offenbar definiert die Totalvariation $\|\mu \|_{Var}$ eine Norm auf $ba(K), ca(K), rba, rca$
	\end{definition}	
	
	Resultat: Für einen Ring $R$, $\mu \in ba(K)$ reellwertig ist 
	$$\mu^+:=\frac{1}{2}(|\mu| + \mu),\qquad \mu^-:=\frac{1}{2}(|\mu| - \mu)$$
	nicht negativ und beschränkt.
	Es ist $\mu = \mu^+ - \mu^-$, diese Zerlegung heißt \textit{Jordan-Zerlegung}.
	Ist $\mu$ regulär, so sind $\mu^+,\,\mu^-$ regulär.
	Für komplexwertiges $\mu$ ist
	$$\mu = Re(\mu)^+ - Re(\mu)^- + i(Im(\mu)^+ -Im(\mu)^-).$$
	Wir betrachten nochmal das (Riemann-) Integral stetiger Funktionen:
	$K$ sei kompakt, $B_0$ wie oben und $\mu : B_0 \to \K$ additiv mit $\|\mu \|_{Var} < \infty$. Für Treppenfunktionen 
	$$f = \sum^k_{i=1} 1_{E_i} a_i,\, k\in\N,\,a_i\in\K,\,E_i\in B_0$$ 
	ist 
	$$\int_K f d\mu := \sum^k_{i=1} a_i \mu(E_i)$$
	unabhängig von der Darstellung von $f$ und es ist offenbar $\abs{\int_K fd\mu} \leq \|f\|_\infty \|\mu \|_{Var}$. 
	Wir zeigen nun, dass sich jedes $f \in C(K)$ durch Treppenfunktionen approximieren lässt.
	
	\begin{thm}[Satz von Riesz-Radon, Dualraum von $C(K)$]
		$K$ kompakt, Hausdorffsch. Durch
			$$ J: rca(K) \to C(K)' \quad \mu \mapsto (f\mapsto \int_K f d\mu)$$
		ist ein isometrischer Isomorphismus definiert
	\end{thm}
	
	\begin{proof}
		Nicht hier!
	\end{proof}

	\begin{bem*}
		\begin{enumerate}[a)]
			\item $x' \in C([0,1])'$, $\fop{f,x'} = f(0) \df \norm{x'} = 1$
			$\mu(E) = \{ 1 \quad 0\in E\quad 0\; sonst$ $\forall E\in\sigma([0,1]$ (Diracmaß). 
			Es gilt $\int_K f d\mu = f(0)$. \todoy %TODO
			\item $g\in L^1([0,1])$, $x' \in C([0,1])'$, $\fop{f,x'} = \int_{[0,1]} f g d\lambda$,
			$\mu := g d\lambda$, $\mu(A) := \int_{A} g d\lambda$ $\df \fop{f,x'} = \int_{[0,1]} f d\mu$
		\end{enumerate}
	\end{bem*}

%%%%%%%%%%%%%%%%%%%%%%%%%%%%%%%%%%%%%%%%%%%%%%%%%%%%%%%%%%%%%%%%%%%%%%%%%%%%%%
	\section{Kompaktheit in $C(K)$ und $L^p$}
%%%%%%%%%%%%%%%%%%%%%%%%%%%%%%%%%%%%%%%%%%%%%%%%%%%%%%%%%%%%%%%%%%%%%%%%%%%%%%

	\begin{thm}[Satz Arzela-Ascoli]
		$K\subseteq \R^n$ kompakt, $M\subseteq C(K)$. Dann gilt 	
			$$ M \text{ präkompakt} \aq M \text{ beschränkt} \text{ und } \text{gleichgeradig stetig}$$
		Dabei heißt eine Menge $M$ gleichgeradig stetig, falls 
			$$ \sup_{f\in M} |f(x) - f(y)| \to 0 \text{ für } x,y \in K \text{ mit } x - y \to 0$$
			$$ (d.h. \forall \epsilon \exists \delta > 0: \forall x,y \in K \text{ mit } |x-y| < \delta \text{ und } \forall f\in M: |f(x) - f(y)| < \epsilon$$
	\end{thm}

	\begin{proof}
		\todor
	\end{proof}

	\begin{thm}[Präkompaktheit in $L^p$]
		$p\in[1,\infty)$, $M \subseteq L^p(\R^n)$. Dann sind äquivalent
		\begin{enumerate}[(i)]
			\item $M$ präkompakt
			\item a) $M$ beschränkt und 
				b) $\sup_{f\in M} \norm{f(\cdot + h) - f(\cdot)}_p \to 0$ bei $h\to 0$
				c) $\sup_{f\in M} \norm{1_{\R^n\backslash U_r(0)}f}\to 0$ für $r\to 0$
		\end{enumerate}
	\end{thm}
	\begin{proof}
		hier nicht, siehe bspw. Alt	
	\end{proof}

	\begin{bem*}[Kompaktheit in $L^p(\Omega)$]
		Fasse $L^p(\Omega)$ als UVR von $L^p(\R^n)$ auf via
			$$ L^p(\Omega) \ni f \mapsto 1_\Omega f \in L^p(\R^n)$$
	\end{bem*}

%%%%%%%%%%%%%%%%%%%%%%%%%%%%%%%%%%%%%%%%%%%%%%%%%%%%%%%%%%%%%%%%%%%%%%%
	\section{Sobolevräume}
%%%%%%%%%%%%%%%%%%%%%%%%%%%%%%%%%%%%%%%%%%%%%%%%%%%%%%%%%%%%%%%%%%%%%%%

		\begin{lemma}[partielle Integration]
			$\Omega \subset \R^n$ offen, beschränkt und habe stückweise glatten Rand. Dann gilt für $u,v \in C^1(\ov{\Omega})$.
				$$\int_\Omega v(x) \frac{\partial}{\partial x_i} u(x) dx = 
				\int_{\partial \Omega} v(x)u(x) e_i^T n(x) ds(x) - \int_\Omega \frac{\partial}{\partial x_i} v(x) u(x) dx$$
				wobei n der Einheitsnormalenvektor ist und das erste Integral das Hyperflächenintegral.
		\end{lemma}
		\begin{proof}
			\todor
		\end{proof}

	\begin{bem}[Notation]
		$\alpha \in \N_0^n:$ Multiindex, $D^\alpha f := \frac{\partial^|\alpha|}{\partial x^\alpha}f$ Ableitung
	\end{bem}

	\begin{bem}
		$f\in C^m(\Omega), \phi\in C_0^\infty(\Omega) = \{f\in C^\infty(\Omega): suppf\subset \circ\Omega\}.$ Anwendung von Lemma 4.12. liefert 
		 $$ \int_\Omega D^\alpha f(x) \phi(x) dx = (-1)^{|\alpha|}\int_\Omega f(x) D^\alpha \phi(x) dx$$ 
	\end{bem}

	\begin{definition}[schwache Ableitung]
		$\Omega \subset \R^n$ offen, $f\in L^1_{\text{loc}}(\Omega), \alpha \in N^n_0$. Wenn ein 
		$w \in L^1_{\text{loc}}(\Omega)$ existiert, so dass 
			$$\int_\Omega f D^\alpha \varphi d\lambda^n = (-1)^|\alpha| \int_\Omega w \varphi d\lambda^n 
			\quad \forall \varphi \in C_0^\infty(\Omega)$$
			dann heißt w \textit{schwache Ableitung} von $f$. $f \in L^1_{\text{loc}}(\Omega)$ heißt \textit{m-mal schwach diffbar}, wenn die $\alpha-ten$ schwache Ableitung exisiert $\forall |\alpha| \leq m$. 
	\end{definition}
	
	\begin{bem*}[Lemma (Übung)]
		Sei $f\in L^1_{\text{loc}}$, so dass 
			$$\int_\Omega f \phi d\lambda^n = 0 \; \forall \phi \in C^\infty_0(\Omega) \aq f = 0$$
	\todog[Übungsbeweis] Beweisen?
	\end{bem*}

	\begin{thm}
		Vor. wie in Def. 4.15. Dann gilt: Falls existent, so ist die $\alpha-te$ schwache Ableitung von $f$ eindeutig bestimmt. 
	\end{thm}

	\begin{proof}
		\todor	
	\end{proof}

	\begin{bsp}
		\begin{enumerate}[a)]
			\item \todoo[Beispiel vervollständigen]
			\item
			\item Ist $f$ $\alpha$-mal stetig differenzierbar, so ist $f$ $\alpha$-mal schwach differenzierbar, und die konventionelle Ableitung stimmt mit der schwachen Ableitung überein. (Lemma 4.12)
		\end{enumerate}
	\end{bsp}

	\begin{definition}[Sobolevräume]
		$\Omega \subset \R^n$ offen, $K\in\N_0$	
		\begin{enumerate}
			\item $p\in [1,\infty)$, $W^{k,p}(\Omega) = \set{f\in L^p(\Omega): D^\alpha f\in L^p(\Omega) \forall |\alpha| \leq k}$ mit Norm $\norm{f}_{W^{k,p}} = (\sum_{|\alpha| \leq k} \norm{D^\alpha f}^p_{L^p(\Omega)})^{\frac{1}{p}}$
		\item $(p = \infty)$ $W^{k,\infty}(\Omega) = \set{f\in L^\infty(\Omega): D^\alpha f \in L^\infty(\Omega) \; \forall |\alpha| \leq k}$ mit Norm $\norm{f}_{W^{k,\infty}} = \max_{|\alpha| \leq} \norm{D^\alpha f}_{L^\infty(\Omega)}$
		\end{enumerate}
	\end{definition}

	\begin{thm}
		Vor. wie in Definition 4.18. Dann gilt 
			\begin{enumerate}[a)]
				\item $W^{k,p}(\Omega)$ Banachraum $\forall p\in[1,\infty], k\in \N_0$
				\item $\forall k\in \N_0, p\in [1,\infty): C^\infty(\Omega) \subset W^{k,p}(\Omega)$ ist dicht in $W^{k,p}(\Omega)$.
				\item $p =2$. Dann ist $W^{k,2}(\Omega)$ Hilbertraum mit $(f,g)_{W^{k,2}(\Omega)} = \sum_{|\alpha| \leq k} (D^\alpha f, D^\infty g)_{L^2(\Omega)}$
			\end{enumerate}
	\end{thm}

	\begin{proof}
		siehe Alt.
	\end{proof}

	\begin{definition}[$W^{k,p}_0(\Omega)$]
		$\Omega \subset \R^n$ offen, $k\in \N_0$, $p\in [1,\infty)$.
			$$ W^{k,p}_0(\Omega) := \ov{C^\infty_0(\Omega)}\subset W^{k,p}(\Omega)$$ 
			Wobei Abschluss bezüglich $\norm{\cdot}_{W^{k,p}(\Omega)}$
	\end{definition}
%%%%%%%%%%%%%%%%%%%%%%%%%%%%%%%%%%%%%%%%%%%%%%%%%%%%%%%%%%%%%%%%%
	\section*{Anwendung an elliptischer Randwertprobleme}
%%%%%%%%%%%%%%%%%%%%%%%%%%%%%%%%%%%%%%%%%%%%%%%%%%%%%%%%%%%%%%%%%
	\todor[Komplette Herleitung zu Dirichlet Randwertproblem] hier vieles 
	\begin{lemma}[Poincare-Ungleichung]
		Ist $\Omega \subset \R^n$ offen und beschränkt, so existiert $c > 0$, so dass
			$$ \int_\Omega |u(x)|^2 dx \leq \int_\Omega \norm{grad u(x)}^2dx \quad \forall u\in W_0^{1,2}(\Omega)$$
	\end{lemma}
	
	\begin{proof}
		\todor
	\end{proof}

	\begin{thm}
		$\Omega$ offen und beschränkt. Dann ist $a(\cdot,\cdot)$ ein Skalarprodukt auf $W_0^{1,2}(\Omega).$
	\end{thm}

	\begin{proof}
	\todor
	\end{proof}
zu b)
	\begin{thm}
		$\Omega \subset \R^n$ offen und beschränkt. Dann ist $\Sobolov(\Omega)$ Hilbertraum mit $\norm{\cdot} = a(\cdot,\cdot)$ mit 
		$$a(u,v) = \int_\Omega(grad u)^T \ov{grad v} d\lambda^n$$
	\end{thm}

	\begin{proof}
		\todor	
	\end{proof}

zu c)
	\begin{thm}
		$\Omega \subset \R^n$ offen und beschränkt, $f\in L^2(\Omega)$. Dann ist 
			$$ F: \Sobolov(\Omega) \to \K, \varphi \mapsto \int_\Omega f(x) \varphi(x) dx$$
		in $\Sobolov(\Omega)'$.
	\end{thm}

	\begin{proof}
		\todor	
	\end{proof}

	\begin{cor}
		Unter den gegebenen Voraussetzung existiert ein eindeutiges $u\in \Sobolov(\Omega)$, so dass $a(\varphi, u) = F(\varphi)$ $\forall \varphi \in C^\infty_0(\Omega)$.
	\end{cor}

	\begin{proof}
		\todor	
	\end{proof}

	\begin{bem}[Neumann-RWP]
		$div(A(x)grad u(x)) = f$ auf $\Omega$. mit $n^T A(x) grad u(x) = g$ auf $\partial \Omega$	mit $n^T$ ist Einheitsnormalenvektor. Kann mit ähnlichen Methoden behandelt werden.
		$$\int_\Omega \varphi(x) div (A(x) grad u(x)) dx = - \int_\Omega (grad \varphi(x))^T A(x) gradu(x) dx + \int_{\partial\Omega} \varphi(x) n^T(x) A(x) grad u(x) dx$$ 
	$$ = -a(\varphi, u) + \int_\Omega \varphi g d\lambda^{n-1} = \int_\Omega \varphi(x) f(x) dx$$
	\end{bem}
