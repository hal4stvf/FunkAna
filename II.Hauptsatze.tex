\chapter{Einige Hauptsätze aus der Funktionalanalysis}			
\section{Satz von der offenen Abbildung, Satz vom abgeschlossenen Graphen, Satz von der stetigen Inversen}

%Definition 2.1
	\begin{definition}[offene Abbildung]
	\label{def:2.1}
		Seien $X,Y$ topologische Räume. $f: X\to Y$ heißt \textit{offen}, falls für alle offenen $U\subseteq X$, $f(U)\subseteq Y$ offen ist.
	\end{definition}

%Satz 2.2
	\begin{thm}[Stetigkeit der Inversen unter offenen Abbildung]
	\label{thm:2.2}
		Es seien $X,Y$ topologische Räume und $f: X\to Y$ injektiv. Dann sind äquivalent
		\begin{enumerate}[(i)]
			\item $f: X \to f(X)$ offen (Relativtopologie von $Y$ auf $f(x)$)	
			\item $\inv{f}: f(X) \to X$ stetig.
		\end{enumerate}
	\end{thm}

	\begin{proof}
		\begin{itemize}[]
			\item $"'(i) \df (ii)"':$ Sei $U \subseteq X$ offen, dann ist $ \inv{(\inv{f})}(U) = f(U)$ offen, also ist $\inv{f}$ stetig.
		\item $"'(ii) \df (i)"':$ Sei $U \subseteq X$ offen, $\inv{f}$ stetig 
			$\df f(U) = \inv{(\inv{f})}(U)$ offen $\df f $ offen.
		\end{itemize}
	\end{proof}

%Lemma 2.3
	\begin{lemma}[Äquivalenzen zu Offenheit]
	\label{lem:2.3}
		Seien $X,Y$ normierte Räume, $T: X\to Y$ ein linearer Operator. Dann sind äquivalent:
			\begin{enumerate}[(i)]
				\item $T$ ist offen.
				\item $\forall r > 0: T(U_r(0))$ ist eine Nullumgebung.
				\item $\exists r > 0 : T(U_r(0))$ ist eine Nullumgebung.
				\item $\exists r > 0 : T(U_1(0))$ ist eine Nullumgebung.
			\end{enumerate}
	\end{lemma}
	\begin{proof}
					Vergleiche \hyperref[B2.3]{ÜA3, Blatt 2.}
	\end{proof}

%Satz 2.4 (Satz von der offenen Abbildung, Satz von Banach-Schauder, openmapping theorem.)
	\begin{thm}[Satz von der offenen Abbildung, Satz von Banach-Schauder, open-mapping theorem]
	\label{thm:2.4}
		Seien $X,Y$ Banachräume und $T\in \BS$ surjektiv. Dann ist $T$ offen.	
	\end{thm}

 	\begin{proof}
 		Wir zeigen, dass $(ii)$ aus \hyperref[lem:2.3]{Lemma 2.3} gilt.
 		\begin{enumerate}[1. {Schritt}]
 			\item Wir zeigen $\exists \varepsilon_0 > 0$, so dass $U_{\varepsilon_0}(0) \subseteq \overline{T(U_1(0))}.$ Weil $T$ surjektiv gilt $Y = \bigcup_{n\in\N} T(U_n(0))$.
 			Da $Y$ Banachraum, so gilt nach Baire
 			$\exists N\in\N: \overline{T(U_N(0))} \neq \emptyset$
 			$\exists y_0 \in \overline{T(U_N(0))}, \varepsilon > 0$, so dass 
 			$\U{\varepsilon}{0} \subseteq \overline{T(U_N(0))}$.
 			Aus $\U{\varepsilon}{0} \subseteq \frac{1}{2} \U{\varepsilon}{y_0} + \frac{1}{2} \U{\varepsilon}{-y_0}$ und 
 			$\overline{T(U_N(0))} = \frac{1}{2} \overline{T(U_N(0))} + \frac{1}{2} \overline{T(U_N(0))} $		
 			folgt $ \U{\varepsilon}{0} \subseteq \overline{T(U_N(0))}$
 			$\df \U{\frac{\varepsilon}{N}}{0} \subseteq \overline{T(U_1(0))}$.
 			$\varepsilon_0 := \frac{\varepsilon}{N}$.
 			\item Wir zeigen $\U{\varepsilon_0}{0} \subseteq {T(U_1(0))}$
 			Sei $y\in \U{\varepsilon_0}{0}$. Wähle $\varepsilon > 0$ mit $ \norm{y} < \varepsilon < \varepsilon_0$, $\overline{y} : = \frac{\varepsilon_0}{\varepsilon} y$ 
 			$\df$ $\norm{\overline{y}} < \varepsilon_0$
 			$\df \overline{y} \in \overline{T(U_1(0))}$
 			$\df \exists y_0 = Tx_0 \in {T(U_1(0))}$ mit $\norm{\overline{y} - y_0} < \alpha \varepsilon_0$, wobei $0 < \alpha < 1$ mit $\frac{\varepsilon}{\varepsilon_0} \cdot \frac{1}{1-\alpha} < 1$
 			Betrachte nun $\frac{\overline{y} - y_0}{\alpha} \in \U{\varepsilon_0}{0}$
 			$\df \exists y_1 = Tx_1 \in {T(U_1(0))}$ mit $\norm{\frac{\overline{y}-y_0}{\alpha} - y_1} < \alpha \varepsilon_0
 			\df \norm{\overline{y} - (y_0 + \alpha y_1)} < \alpha^2 \varepsilon_0$
 			Behandle $\frac{\overline{y} - (y_0 +\alpha y_1)}{\alpha^2}$ mit derselben Methoden,
 			erhalte, $y_2 = Tx_2 \in {T(U_1(0))}$ mit $\norm{\overline{y} - (y_0 + \alpha y_1 + \alpha^2 y_2)} < \alpha^3 \varepsilon_0$
 			Erhalte so induktiv eine Folge $(x_n)$ in $\U{0}$ mit $\norm{\overline{y} - T(\sum_{k=0}^n \alpha^k x_k)} < \alpha^{n+1} \cdot \varepsilon_0$.
 			Weegen $\alpha < 1$ gilt $\overline{x} := \sum_{\alpha=0}^\infty \alpha^k x_k$ konver.
 			$\df[T beschränkt] T\overline{x} = \overline{y}$ Für $x = \frac{\varepsilon}{\varepsilon_0} \overline{x}$
 		gilt $Tx = y$ und  
 		$\norm{x} = \frac{\varepsilon}{\varepsilon_0} \norm{\overline{x}} \leq \frac{\varepsilon}{\varepsilon_0} \sum_{k=0}^\infty \alpha^k \underbrace{\norm{x_k}}_{< 1} < \frac{\varepsilon}{\varepsilon_0} \sum_{k=0}^\infty \alpha^k = \frac{\varepsilon}{\varepsilon_0} \cdot \frac{1}{1-\alpha} < 1$
 		$\df y\in T(U_1(0)).$ Also $\U{\varepsilon}{0} \subseteq T(U_1(0))$.
 		\end{enumerate}
 	\end{proof}

	\todog[ÜA]
	ÜA: $X,Y$ Banachräume, $T\in \BS$ ist offen (relativ in im$T$) $\Leftrightarrow imT$ abgeschlossen. 
		\begin{proof}[Idee]
		$"'\Leftarrow"'$ gilt nach \hyperref[thm:2.4]{Satz 2.4.} ($imT$ abgeschlossen $\df imT$ ist Banachraum)\\
		$"'\df"'$ betrachte injektive Abbildung $\hat{T}: X\setminus KerT \to imT, x+KerT\mapsto Tx$
		\end{proof}

%%%%%%% 2.5
	\begin{thm}[Satz von der stetigen Inversen, inverse mapping theorem]
	\label{thm:2.5}
	$X, Y$ Banachräume. $T\in\BS$ bijektiv $\df \inv{T}\in \B(Y,X)$	
	\end{thm}
	
	\begin{proof}
		Folgt aus \hyperref[thm:2.4]{open mapping thm }und \hyperref[thm:2.2]{Satz 2.2} (wichtig Banachraum!)
	\end{proof}

%%%%%%% 2.6
	\begin{definition}[Graph]
	\label{def:2.6}
		$X,Y$ Mengen, $f: X\to Y$ Abbildung. Der \qmarks{Graph von $f$} ist 
			$G(f) := \{(x,f(x)) : x\in X\} \subseteq X\times Y$
	\end{definition}

%%%%%%% 2.7
	\begin{definition}[Metrik auf kartetischen Produkt]
	\label{def:2.7}
		$X,Y$ metrische Räume. Dann ist auf $X\times Y$ eine Metrik via 
			$d((x_1,y_1),(x_2,y_2)) : (d(x_1,x_2)^2 + d(y_1,y_2)^2)^{\frac{1}{2}}$ definiert. (erhält Parallelogrammgleichung und damit Skalarproduktstruktur)
	\end{definition}

	Beachte 
		\begin{enumerate}[(i)]
			\item $X\times Y$ vollständig $\Leftrightarrow$ $X, Y$ vollständig
			\item $X\times Y$ normierter Raum $\Leftrightarrow$ $X,Y$ nomierte Räume
			\item $X\times Y$ Skalarproduktraum $\Leftrightarrow$ $X,Y$ Skalarproduktraum. 
			\item abgeschlossene Metrik ist äquivalent zu $\max\{d(x_1,x_2),d(y_1,y_2)\}$ und 
			$(d(x_1,x_2)^p + d(y_1,y_2)^p)^{\frac{1}{p}}$, $p\in (1,\infty)$
		\end{enumerate}

%%%%%%% 2.8
	\begin{definition}[Graphenabgeschlossenheit]
	\label{def:2.8}
		$X,Y$ metrische Räume, $f: X\to Y$ heißt \textit{graphenabgeschlossen}, wenn $G(f) \subseteq X\times Y$ abgeschlossen ist.	
	\end{definition}

%%%%%%% 2.9
	\begin{bem}[Bem. zur Graphenabgeschlossenheit]
	\label{bem:2.9}
		\begin{enumerate}
			\item $f$ graphenabgeschlossen $\aq$ $(x_n)$ in $X$ mit $x_n \to x$ und $f(x_1) \to y$ 
			$\df f(x) = y$)
			\item $T: X\to Y$ lineare Operator $\df$ $G(T) \subseteq X\times Y$ UVR.
			\item $f$ stetig $\df$ $f$ graphenabgeschlossen.
			\item Umkehrung von $(iii)$ gilt i.A. nicht: Gegensbeispiel: $f: \R \to \R$ 
				$x\mapsto \begin{cases} 0& x=0\\ \frac{1}{x} & sonst \end{cases}$
		\end{enumerate}
	\end{bem}

%%%%%%% 2.10
	\begin{thm}[Satz vom abgeschlossem Graphen, closed graph theorem]
	\label{thm:2.10}
		$X,Y$ Banachräume, $T: X\to Y$, lineare Operatoren. Dann sind äquivalent:
			\begin{enumerate}[(i)]
				\item $T$ graphenabgeschlossen
				\item $T\in \BS$
			\end{enumerate}
	\end{thm}

	\begin{proof}[TODO]
		\begin{itemize}
			\item \qmarks{$ii\df i$} Klar, weil $\dots$ 
			\item Definiere Abbildung, $S: G(T) \to X, (x,Tx) \mapsto X$
			$\df$ $S$ bijektiv und linear. Wegen 
				$\norm{S(x,Tx)}_X = \norm{X} \leq (\norm{x}_X^2 + \norm{Tx}_Y^2)^{\frac{1}{2}})$
				gilt $S\in \B(G(T),X)$ mit $\norm{S} \leq 1$ .
				Weil $G(T) \subseteq X\times Y$ und $X,Y$ Banachräume, ist $G(T)$ Banachraum.
				$\df[S2.4]$ $\inv{S} \in \B(X, G(T)) \df (\norm{x}_X^2 + \norm{Tx}_Y^2)^{\frac{1}{2}} = \norm{(x,Tx)}_{X\times Y} = \norm{\inv{S}(x)} \leq \norm{\inv{S}} \cdot \norm{x}_X$
				$\df \norm{Tx}_Y \leq (\norm{x}_X^2 + \norm{Tx}_Y^2)^{\frac{1}{2}} \leq \norm{\inv{S}}\cdot\norm{x}_X$
				$\df T\in\BS$
		\end{itemize}
	\end{proof}

%%%%%%% 2.11
	\begin{bem}[Ein paar Anwendungen]
	\label{bem:2.11}	
		\begin{enumerate}
			\item Aus Inverse mapping thm folgt: $(X,\norm{\cdot}_1$, $(X,\norm{\cdot}_2)$ BRe, $\norm{\cdot}_1$ stärker als $\norm{\cdot}_2$ $\df$
				 $\norm{\cdot}_2$ stärker $\norm{\cdot}_1$.
				 \begin{proof}
						\spcm
						Sei $I_X: (X,\norm{\cdot}_1)\to(X,\norm{\cdot}_2) \overset{Vor.}{\df}  I_x$ beschränkt $\overset{IMT}{\df} I_X^{-1}
						: (X,\norm{\cdot}_2) \to (X,\norm{\cdot}_1)$ beschränkt $\df \norm{\cdot}_2$ ist stärker als $\norm{\cdot}_1 \spcm$.
				 \end{proof}
			\item Betrachte $X = C([a,b])$, $Y = C^1([a,b])$ mit Normen $\norm{x}_X = \max_{t\in[a,b]} |x(t) | = \norm{x}_\infty$, $\norm{x}_Y = \norm{x}_\infty + \norm{x'}_\infty$, $X,Y$ Banachräume.\par
			Ist $T\in \B(C([a,b]))$ mit $imT \subset C^1([a,b])$. Dann ist $T\in \B(C([a,b]), C^1([a,b]))$
				\begin{proof}
					$x, x_n \in X y\in Y$ mit $\norm{x_n - x}_X \to 0$, $\norm{Tx_n - y}_Y \to 0$
					$\df \norm{x_n - x}_X \to 0$ und $\norm{Tx_n - y}_X \to 0$, da $\norm{z}_X \leq \norm{z}_Y \; \forall z\in Y$. $\df[T\in\B(X)]$ $\limes \norm{Tx_n - Tx} = 0 \df y = Tx$
					$\df T$ graphenabgeschlossen $\df[X,Y \; BRe]$ $T\in \BS$.
				\end{proof}
		\end{enumerate}

	\end{bem}

%%%%%%% 2.12
	\begin{bem}[Charakterisierung grahpenabgeschlossen]
	\label{thm:2.12}
		$T: X\to Y$ lineare Operatoren. Dann sind äquivalent:
			\begin{enumerate}[(i)]
				\item $T$ graphenabgeschlossen 
				\item $\forall (x_n)$ in $X$ mit $x_n \to 0$, $y\in Y$ mit $ (Tx_1) \to y$ folgt $y=0$ 
			\end{enumerate}

		\begin{proof}
			Nutze Linearität:\\
			"'$\df$"' klar (Spezialisierung auf Nullfolgen)\\
			"'$\Leftarrow$"' angenommen $(x_n) \to x, T(x_n) \to y \df (x_n-x)$ ist Nullfolge.
		\end{proof}

	\end{bem}

	\section{Das Prinzip der gleichmäßigen Beschränkheit, Satz von Banach-Steinhaus}

%%%%%%% 2.13
	\begin{thm}[Satz von Osgood]
	\label{thm:2.13}
		$X$ normierte Raum, $E \subset X$ Teilmenge von 2. Kategorie.	Sei $\mathcal{F} = \{f_\alpha: X\to \R$ stetig, $\alpha \in A\}$ eine Menge von FUnktionen. $\F$ sei auf $E$ punktweise beschränkt, d.h. $\forall x\in E$ $\exists M_x > 0$, so dass $f_\alpha (x) \leq M_x$ $\forall \alpha \in A$. Dann existiert abgeschlossene Kugel $K\subseteq X$, auf der $\F$ glm nach oben beschränkt ist. D.h.
		$$\exists M > 0 \text{ s.d } f_\alpha (x) \leq M \; \forall \alpha \in A, x\in K.$$
	\end{thm}
	
	\begin{proof}
		Für $n\in \N$, def $E_n : \{x\in X: f_\alpha(x) \leq n \forall \alpha \in A\}$
		$= \bigcap_{\alpha\in A} \underbrace{\{ x\in X: f_\alpha (x) \leq n\}}_{\text{abgeschlossen, da f stetig}}$
		$\df E_n$ abgeschlossen. Ferne gilt $E\subset \cup_{n\in\N} E_n$ wegen Annahme (punktweise Beschränkt). $\df[2. Kate] \cup_{n\in\N} E_n$ von 2. Kategoriere 
		$\df$ $\exists n_0 \in \N$ so, dass $\inner{E}_{n_0} = \inner{\overline{E_{n_0}}} \neq \emptyset.$
		$\df$ Für $U = \inner{E}_{n_o} :\sup_{\alpha \in A, x \in U} f_\alpha (x) \leq n_0 =: M$
		Insbesondere $\exists x_0 \in U, \delta > 0$, so dass $K:= \overline{U_\delta(x_0)} \subseteq U$. Dann gilt $\forall \alpha \in A, x\in \overline{U_\delta(x_0)}: f_\alpha (x) \leq M.$
	\end{proof}

%%%%%%% 2.14
	\begin{cor}[Prinzip der glm Beschränkheit]
	\label{cor:2.14}
		$X,Y$ normierter Räume, $E\subset X$ von 2. Kategorie, 
			$\F \subseteq \BS$ mit $\forall x \in $ $\exists M_x > 0$ so dass $\norm{Tx} \leq M_x \forall T \in \F$.
			Dann gilt: $$\exists M > 0 \text{ so dass } \norm{T} \leq M \forall T \in \F$$
	\end{cor}

	\begin{proof}
		Für $T \in \F$ definiere $f_T: X \to \R, x \mapsto \norm{Tx} \df f_T$ ist stetig $\forall T \in \F, \{f_T: T \in \F\}$ ist pw. beschränkt $\overset{Osgood}{\df} \exists M,r >0, x_0 \in E: \disp \sup_{\underset{T \in \F}{x \in \overline{U_r(x_0)}}} \norm{Tx} \in M$. Für $x \in U_1(0)$ gilt dann
		$$\norm{Tx}=\norm{\frac{1}{r}T(rx+x_0)-\frac{1}{r}Tx_0} \leq \frac{1}{r} \underbrace{\norm{T(rx+x_0)}}_{\leq M} + \frac{1}{r} \underbrace{\norm{Tx_0}}_{\leq M} \leq \frac{2M}{r} \df \norm{T} \leq \frac{2M}{r} ~\forall T \in \F.$$
	\end{proof}

%%%%%%% 2.15
	\begin{cor}[beschränktheit von bestimmten Operatoren]
	\label{cor:2.15}
		$X$ Banachraum, $Y$ normierter Raum. Sei $\F \subseteq \BS$, so dass $\forall x \in X$ $\exists M_x > 0: \norm{T_x} \leq M_x \; \forall T\in \F$. Dann existiert ein $M > 0$, so dass $\norm{T} \leq M$ $\forall T \in \F$.
	\end{cor}

	\begin{proof}
		$X$ Banachraum $\df[Baire]$ $X$ von 2. Kategorier. Resultat folgt aus Kor. 2.14.
	\end{proof}

%%%%%%% 2.16
	\begin{cor}[Unbeschränkte Operatoren]
	\label{cor:2.16}
		$X$ Banachraum, $Y$ normierter Raum, $\F \subseteq \BS$, so dass 
			$$ \sup_{T\in\F} \norm{T} = \infty \vspace{-1em} $$. 
		Dann gilt 
			\begin{enumerate}[(i)]
				\item $\exists x_0 \in X$: $\sup_{T\in \F} \norm{Tx_0} = \infty$
				\item Die Menge $\{x_0 \in X: \sup_{T\in \F} \norm{Tx_0} = \infty\}$ ist dicht.
					\begin{proof}
						Angenommen $Z \subseteq X$ nicht dicht $ \df \exists r>0, x\in X: $
							$$ \overline{\U[r]{x_0}} \subseteq X \backslash Z \df \forall x \in \overline{\U[r]{x_0}} : \sup_{T\in\F} \norm{Tx} < \infty$$
							$\df[2.14] \sup{T\in \F} \norm{T} < \infty$ Widerspruch!
					\end{proof}
			\end{enumerate}
	\end{cor}

TODO: Ein Beispiel für starke Konvergenz aber keine Konvergenz von Operatoren oder sowas.

%%%%%%% 2.17
	\begin{thm}[Aussagen zu punktweiser Konvergenz]
	\label{thm:2.17}
		$X$ Banachraum, $Y$ normierter Raum. $(T_n)$ Folge in $\BS$, so dass 
			$$ Tx = \limes T_n x \; \forall x\in X.$$ 
		Dann gilt $T\in \BS$, $\{\norm{T_n} : n\in \N\}$ beschränkt und $\norm{T} \leq \limes\inf \norm{T_n}$.
	\end{thm}

	\begin{proof}
		Linearität von $T$ ist klar. Weiter gilt 
			$$ \limes |\norm{T_nx}-\norm{Tx}| \leq \limes \norm{T_nx-Tx} = 0 ~\forall x \in X \df ~\forall x \in X ~\exists M_x>0 \text{, sodass } \sup_{n \in \N} \norm{T_nx}\leq M_x$$ 
		$\df \exists M>0,$ so dass $ \disp \sup_{n \in \N} \norm{T}\leq M<\infty$. 
		Für $x \in X$ und jede TF $(X_{n_k})$, sodass $\norm{T_{n_k}}$ konvergiert, gilt
			$$ \norm{Tx}=\lim_{k \to \infty}\norm{T_{n_k}x}\leq \lim_{k \to \infty}\norm{T_{n_k}}\norm{x} \df \norm{Tx} \leq \liminf_{n \to \infty} \norm{Tn}\norm{x} \df \norm{T} \leq \liminf_{n \to \infty} \norm{T_n}.$$ 
	\end{proof}

%%%%%%% 2.18
	\begin{thm}[Satz von Banach-Steinhaus]
	\label{thm:2.18}
		$X,Y$ Banachräume, $(T_n)$ Folge in $\BS$. Dann konvergiert $(T_n)$ punktweise gegen ein $T\in \BS$, genau dann wenn folgende beiden Bedingungen erfüllt sind:
			\begin{enumerate}[(1)]
				\item $\exists M > 0$, so dass $\norm{T_n} \leq M \; \forall n\in \N$
				\item $\exists D \subset X$ dicht, so dass $(T_n x)$ CF $\forall x\in D$.
			\end{enumerate}
	\end{thm}

	\begin{proof}
		"'$\df$"' 1. folgt aus \hyperref[thm:2.17]{Satz 2.17.}, 2. ist klar (nehme $D=X$)\\
		"'$\Leftarrow$"' Sei $x \in X, \varepsilon>0 \df \exists y \in D: \norm{x-y}<\frac{\varepsilon}{3M}, \exists N \in \N: \norm{T_ny-T_my}<\frac{\varepsilon}{3} ~\forall n,m \leq N$. Dann gilt 
\begin{align*}
		\forall n,m \leq N: \norm{T_mx-T_nx} &\leq \norm{T_mx-T_my} + \norm{T_my-T_ny} + \norm{T_ny-T_nx}\\ &\leq \norm{T_m} \norm{x-y}+ \norm{T_my-T_nx} + \norm{T_n} \norm{x-y}\\ &\leq M \frac{\varepsilon}{3M} + \frac{\varepsilon}{3} + \frac{\varepsilon}{3M} = \varepsilon.
		\end{align*}
		$\df (T_nx)$ ist Cauchyfolge in $Y \df (T_nx)$ konvergiert in $Y 
		\overset{\hyperref[thm:2.17]{2.17.}}\df T \in B(X,Y).$ 
	\end{proof}
