\chapter{Uebungen}
\section{ Blatt 1}
	
\paragraph{\underline{Aufgabe 1}}
Es sei $(X, \|\cdot\|)$ ein normierter Raum.
\begin{beh}
Es ist äquivalent:
\begin{enumerate}[(a)]
	\item	$\sum^\infty_{n=0} \|u_n\|<\infty\df$ Die Reihe $\sum^\infty_{n=0}u_n$ konvergiert in $X$.
	\item	$X$ ist ein Banachraum.
\end{enumerate}
\end{beh}
\begin{proof}
  a$\df$b:
	Da X bereits ein normierter Raum ist, ist nur noch zu zeigen, dass X vollständig ist. Sei $(u_n)$ eine Cauchyfolge in $X$, wobei o.B.d.A $u_0=0$, sonst betrachten wir $(z_n)$ mit $z_{n+1}:=u_n,\,z_0:=0$ statt $(u_n)$. Wir definieren: $s_k:= \sum^k_{n=0}u_n-u_{n-1} = u_k$ (Teleskopsumme).\par
	Unser Ziel ist es, die Reihe $s_k$ \glqq geschickt\grqq{} mit $x_n := \frac{1}{n^2}$ zu majorisieren.\\
	Da $u_n$ eine Cauchyfolge ist, gibt es für jedes $x_n$ ein (kleinstes) $N_n:=N(x_n)\in\N$, so dass $\| u_{N(x_n)}-u_m\| \leq x_n\; \forall m \geq N_n$.
	Das Bemerkenswerte an $(N_n)$ ist, dass diese Folge monoton steigt. 
	Es ist nun für hinreichend großes $k$ und geeignetes $n'\in\N$ mit der Dreiecksungleichung und der Eigenschaft der Teleskopsumme:
	$$\|s_k\| =\| \sum^k_{n=0}u_n-u_{n-1}\|\leq \underbrace{\|u_1 - u_0 \| + \| u_2 - u_1\| + \dots }_{=:x_0}+ \underbrace{\|u_{N_2} - u_{N_1}\|}_{\leq x_1} + \underbrace{\|u_{N_3} - u_{N_2}\|}_{\leq x_2} + \dots + \underbrace{\|u_k-u_{N(n')}\|}_{\leq x_{n'}}$$
	Bzw.:
	$$\|s_k\| = \|\sum^k_{n=1} u_{N_n} - u_{N_{n-1}} \| \leq \sum^{n'}_{n=0} x_n\text{ mit } u_{N_0}:=u_0
	$$
	Es ergibt sich also: 
	$$\|s_k\| \leq \sum^{n'}_{n=0}x_n \leq \sum^k_{n=0}x_n \df \lim_{k\to\infty} \|s_k\| \leq \sum^\infty_{n=0}x_n < \infty$$ 
	Und mit der Eigenschaft a) ist nun: $u_k = s_k \longrightarrow \alpha \in X$. Also konvergiert $u_k$ in $X$, damit ist $X$ ein Banachraum.\par
 b$\df$a:
 	Sei X ein Banachraum und $(u_n)$ eine Folge in $X$, so dass $\sum^\infty_{n=0}\|u_n\| < \infty$, dann ist dank der Dreiecksungleichung: $\| \sum^\infty_{n=0}u_n \| \leq  \sum^\infty_{n=0}\|u_n\| < \infty$.\\
 	Dank der Vollständigkeit von $X$ können wir das Majorantenkriterium (der Beweis zu der Gültigkeit dieses Kriteriums, ist analog zum reellen Fall) anwenden, und somit konvergiert die Reihe $\sum^\infty_{n=0}u_n$.\par
 	
 	Nachtrag:\\
 	Wesentlich eleganter geht dieser Beweis mithilfe des Cauchykriteriums für Reihen. Die Idee ist aber im Grunde die gleiche.
\end{proof}

\paragraph{\underline{Aufgabe 2}}
Es seien $(X,\|\cdot\|_X)$ und $(Y,\|\cdot\|_Y)$ normierte Räume und $T\in \B(X,Y)$.
\begin{beh}
	Es gilt:
	$$\|T\|:=\sup_{x\in U_1(0)} \|Tx\|_Y = \sup_{x\in\overline{U_1(0)}}\|Tx\|_Y = \sup_{x\in \partial U_1(0)} \|Tx\|_Y = \sup_{x\in X\backslash \{0\}} \frac{\|Tx\|_y}{\|x\|_X}.$$
\end{beh}
\begin{proof}
	Aus Satz 1.7 folgt, dass $T$ stetig ist. \par 
	Dank der Definition von $\sup$ und der Stetigkeit von $T,\|\cdot \|_Y$ folgt: $\displaystyle \|T\|=\sup_{x\in\overline{U_1(0)}}\|Tx\|_Y$ \par 
	Nun ist: 
	\begin{equation*}
\begin{split}
\sup_{x\in\overline{U_1(0)}}\|Tx\|_Y & = \sup_{x\in\overline{U_1(0)}\backslash\{0\}}\|\|x\|_X T\left(\frac{x}{\|x\|_X}\right)\|_Y
\\ & = \sup_{x\in\overline{U_1(0)}\backslash\{0\}}\|x\|_X \|T\left(\frac{x}{\|x\|_X}\right)\|_Y
\\ & = \underbrace{\sup_{x\in\overline{U_1(0)}\backslash\{0\}}\|x\|_X}_{=1}
	   \cdot \sup_{x\in\overline{U_1(0)}\backslash\{0\}} \|T\left(\frac{x}{\|x\|_X}\right)\|_Y
\\ & = \sup_{x\in \partial U_1(0)} \|Tx\|_Y  =
	\sup_{x\in X\backslash\{0\}} \|T\left(\frac{x}{\|x\|_X}\right)\|_Y
 = \sup_{x\in X\backslash\{0\}} \frac{\|Tx\|_Y}{\|x\|_X}
\end{split}
\end{equation*}
Wobei wir genuzt haben, dass $T0 = 0$ (für das erste Gleichheitszeichen) und $\|x\|_X,\|Tx\|_Y\geq 0\;\forall x\in X$ (für das dritte Gleichheitszeichen). 
\end{proof}

\paragraph{\underline{Aufgabe 3}}
Es seien $(X, \| \cdot \|_X)$ und $(Y, \| \cdot \|_Y)$ normierte Räume.
\begin{enumerate}[(a)]
\item

\begin{beh}
$$dim(X) < \infty \text{ und } T:X\to Y \text{ linear}\df T \in \B(X,Y)$$
\end{beh}
\begin{proof}
Da $dim(X) <\infty$ und $T$ linear ist, lässt sich $T$ eindeutig als Matrix darstellen. Dann sind aber die Bedingungen aus Bsp. 1.6 a) erfüllt. Damit ist $T \in \B(X,Y)$.\par
Nachtrag:\\
Der Beweis ist noch nicht ganz vollständig. Es ist noch zu zeigen, dass die Koordinatenfunktion in beide Richtungen (also $\phi:X\to Y$ und $\phi^{-1}:Y\to X$, für die Koordinatenfunktion $\phi$ zu einer Basis) stetig ist. 
Dann kann man aber wirklich die Zeilensummennorm betrachten:
$$\forall x\in \EK \text{ gilt } \|Tx\|\leq a\|Tx\|_{\infty} \leq a\|T\|_{\infty} \|x\| \leq a\|T\|_{\infty} < a\infty.$$
Wobei $a\in\R^+$ die geeignet gewählte nur von der Norm abhängige Konstante ist, die von der Äquivalenz zu jeder anderen Norm entsteht.

\end{proof}

\item

\begin{beh}
$$dim(X)<\infty \text{ und } T\in\B(X,Y) \df \|T\|= \max_{x\in\overline{U_1(0)}}\|Tx\|_Y$$
\end{beh}
\begin{proof}
Aus Aufgabe 2 ist bekannt, dass $\displaystyle\sup_{x\in U_1(0)} \|Tx\|_Y = \sup_{x\in\overline{U_1(0)}}\|Tx\|_Y$. Da $\overline{U_1(0)}$ kompakt \textit{(ist es kompakt? Es könnte ja sein, dass X nicht vollständig ist. Müsste aber, denn wir haben ja immernoch den Abschluss... oder?) } ist, und $T,\,\|\cdot\|_Y$ stetig dank Satz 1.7 und Komposition von stetigen Funktionen und da stetige Funktionen auf kompakten Mengen ihr Maximum annehmen, ist die Behauptung bewiesen.\par
Nachtrag:\\
Tatsächlich ist $\overline{\EK}$ in einem endlich dimensionalen Vektorraum über $\R,\C$ stets kompakt. Angeblich gilt sogar \textit{(habe ich das richtig mitbekommen??)}: 
$$\overline{\EK}\subset X \text{ kompakt } \Leftrightarrow dim(X) < \infty$$
\end{proof}

\item

\begin{beh}
Im allgemeinen ist die Aussagen von b) falsch.
\end{beh}
\begin{proof}
Wir geben ein Beispiel für $T\in\B(X,Y)$ derart an, dass 
$$\|T\| \not\in\left\lbrace\|Tx\|_Y|\,x\in\overline{U_1(0)}\right\rbrace.$$
Sei  $$X=Y=\ell^p(\K)=:\ell^p \text{ (beschr. Folgenraum) und } T:\ell^p\to \ell^p,\;(x_n)\mapsto \left(\left(1-\frac{1}{n}\right)x_n\right)$$
Mit der Norm $\displaystyle \|(x_n)\|=\left(\sum^\infty_{n=1} |x_n|^p\right)^{\frac{1}{p}}$\par
Dann ist $\overline{U_1(0)} = \left\lbrace (x_n) \in \ell^p |\,\sum^\infty_{n=1} |x_n|^p \leq 1 \right\rbrace$\par
$T$ ist ein beschränkter linearer Operator, denn:
\begin{enumerate}[(i)]
\item Seien $(x_n),(y_n)\in \ell^p,\,\alpha \in \K\df T(\alpha x_n + y_n) = \left(\left(1-\frac{1}{n}\right)(\alpha x_n + y_n) \right) = \alpha Tx_n + Ty_n$

\item Sei $x_n \in \overline{U_1(0)}$
\begin{equation*}
\begin{split}
 \df 
 \|T(x_n)\|^p 
 = \|\left(\left(1-\frac{1}{n}\right)x_n\right)\|^p 
 & = \sum^\infty_{n=1} \left(\left(1-\frac{1}{n}\right)x_n\right)^p 
 \\ & = \sum^\infty_{n=1} | x_n-\frac{x_n}{n}|^p 
 \\ & \leq \sum^\infty_{n=1} |x_n|^p+|\frac{x_n}{n}|^p
 \\ & \leq \sum^\infty_{n=1} 2|x_n|^p < \infty
 \end{split}
\end{equation*}


\end{enumerate}
Mit der Folge (von Folgen) $i_n$, die an der $n.$ Stelle eine 1 hat, und sonst nur Nullen, ist dann: $T(i_n) \longrightarrow 1$ für $n\longrightarrow \infty$. Dies ist mit Elementen aus $\overline{U_1(0)}$ nicht möglich (Warum?).\par
Nachtrag:\\
Geschickter ist es den Folgenraum $M\subset \ell^p$ einzuschränken, so dass für jedes Element aus $M$ gilt, dass nur endlich viele Folgenglieder ungleich null sind. Der Rest folgt flott.

\end{proof}

\end{enumerate}


	\setcounter{section}{7}
\section{Blatt 8}				

	\begin{beh}
		$X$ normierter Raum. $x' : X\to \R$ linearer Funktional und $ker x'$ abgeschlossen. 
		$$\Rightarrow x' \text{ ist beschränkt}$$
	\end{beh}
	\begin{proof}
		Betrachte $ X/ker x':= \{ \hat{x} := x + N: x\in X\} $ und die lineare Abbildung
			$$ \hat{H} : X/ker \:x' \to \R, \; \hat{H}\hat{x} := x'(x).$$
		Weil $x'$ linear ist und $\hat{H}$ injektiv und $\hat{H}(X/ker\: x') = x'(X)$, folgt
			$$X/ker\:x \text{ und } x'(X) \text{ isomorph zueinander} \quad und \quad \dim(X\ker \:x') = \dim(\R) = 1.$$ 
	Nun ist ein linearer Operator von einem endlich dimensionalen Raum in einen beliebigen normierten Raum stetig. Und damit ist $\hat{H}$ stetig. Nun wird gezeigt, dass daraus auch folgt, dass $x'$ stetig sein muss.\\
	Sei $\hat{x}$ die Restklasse von $x$, dann gilt 
		$$\norm{x'(x)} = \norm{\hat{H}\hat{x}} \leq \norm{\hat{H}} \cdot \norm{\hat{x}} \leq \norm{\hat{H}} \cdot \norm{{x}} \leq M \norm{x}. $$ 
	Mit der in Blatt 4 Aufgabe 1 eingeführt Norm auf Quotientenräume.	
	\end{proof}
