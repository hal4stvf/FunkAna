\newglossaryentry{hausdorffsch}{
	name = {Hausdorffsch, Hausdorffeigenschaft},
	description = {Eine Menge heißt \textit{hausdorffsch}, wenn je zwei versch. Punkte stets disjunkte Umgebungen haben. Metrische Räume sind zum Beispiel hausdorffsch, da zwei versch. Punkte stets einen Abstand $> 0$ haben. Für ein Gegenbeispiel $\nearrow$ topologischer Raum}
		}
	
\newglossaryentry{ess beschrk}{
	name={essentiell beschränkt},
	description={$(\Omega, \hA,\mu)$ sei ein Maßraum. Eine Funktion $f: \Omega \rightarrow \R$ heißt essentiell beschränkt, falls 
						$$\underset{x\in\Omega}{\ess \sup} |f(x)| := {\inf_{\underset{\mu(N)=0}{N \in \hA}} }  
						\sup_{x\in \Omega\backslash N} |f(x)| < \infty$$
						oder auch: f ist fast überall beschränkt. 
									Ein Beispiel ist $f(x) : = x\cdot \chi_\Q(x)$ und $\mu = \lambda$, da $f$ nur auf $\Q$ nicht null ist, und $\Q$ ist Lesbesgue-Nullmenge. }
		}
	
\newglossaryentry{topologisch}{ 
			name = {topologischer Raum},
			description = {$(X,\tT)$ - Sei $X$ eine Menge und $\tT\subseteq P(X)$. Die Elemente von $\tT$ sind die \textit{offenen Mengen}. $\tT$ definiert eine \textit{Topologie}, wenn folgende Eigenschaften erfüllt sind:
		\begin{enumerate}[(i)]
			\item $\emptyset,\,X\in \tT$ 
			\item $A_i\in\tT$ für $i\in I$, $\N \supset I$ endlich $\df$ $\cap_{i\in I} A_i\in\tT$ 
			\item $A_i\in\tT$ für $i\in I$, $I$ bel. Indexmenge $\df$ $\cup_{i\in I}A_i \in \tT$
		\end{enumerate}
		$(X,\tT)$ ist der \textit{topologische Raum}.\\
		Ein Beispiel, für einen topologischen Raum sind die metrischen Räume $(X,d)$: $d$ induziert dann eine Topologie auf $X$, die offenen Mengen sind nämlich durch $d$ bestimmt.\\
		Sei $M:=\{1,2\},\dots$
		\begin{itemize}
			\item $\tT: = \{\emptyset,M\}$. Die triviale Topologie, nur $\emptyset$ und $M$ sind offen.
			\item $\tT:=P(M)$. Die diskrete Topologie, alle Mengen sind offen. Die diskrete Metrik induziert genau diese Topologie.
			\item $\tT:=\{\emptyset,\{1\},\{1,2\}\}.$ M ist hier nicht \glslink{hausdorffsch}{hausdorffsch}, denn egal welche Umgebung man um 2 betrachtet, man kann nicht erreichen, dass 1 nicht in der gleichen ist.
		\end{itemize}
}}

\newglossaryentry{stetig}{
	name = {stetig},
	description = {
		hier alle sache wegen Stetigkeit
		}
	}

\newglossaryentry{Konvergenz}{
  name={Konvergenz},
  description = {}	
}

\newglossaryentry{glm stetig}{
	name = {gleichmäßige Stetigkeit},
	description = {
		hallo		
	},
	sort = stetig
}

\newglossaryentry{glm konvergenz}{
	name = {gleichmäßige Konvergenz von Funktionenfolgen},
	description = {
		Sei $x_k : T \to \R$\\
		Ein Funktionsfolge $\ff{x}{k}{\N}$ \textit{konvergiert gleichmäßig} auf $T$ gegen $x$, wenn
			$$ \forall \varepsilon > 0 \exists k_0 = k_0(\varepsilon): \forall k \geq k_0 \; 
					\forall t\in T : | x_k(t) - x(t) | < \varepsilon$$
			Man schreibt dafür $f_n \to f$ gleichmäßig auf $T$.\\
		Dazu äquivalent ist die Beschreibung durch die Supremumsnorms:
			$$ \forall \varepsilon > 0 \exists k_0 = k_0(\varepsilon): \forall k \geq k_0:
					\| x_k(t) - x(t) \|_{\infty} := \sup_{t\in T} |x_k(t) - x(t)| < \varepsilon$$
			$$\aq \norm{x_k - x}_\infty \to 0 $$
		\begin{proof}
			\begin{description}
				\item[$\Leftarrow$:] 
				\item[$\Rightarrow$:] \footnotesize Sei $\varepsilon > 0$, $k_0 = k_0(\varepsilon)$, so dass $\forall k > k_0$ und $\forall t\in T$ ist. Dann ist für dieses $k$ somit $y_k := x_k - x$ auf $T$ beschränkt und 
					$$ \norm{y_k}_\infty = \norm{x_k - x}_\infty = \sup_{t\in T} |x_k(t) - x(t)| \leq \varepsilon,$$
				 	damit konvergiert die Zahlenfolge $B(T) \ni (\norm{y_m}_\infty, \norm{y_{m+1}}_\infty, \ldots) \to 0$ für $m\to \infty$.\\
				\scriptsize
				Was ausgeschrieben ja genau heißt:
				$$ \forall \varepsilon > 0 \exists k_0 \in \N: \forall k > k_0: |\|g_k\|_\infty - 0| = \|g_k\|_\infty < \varepsilon$$ \normalsize
			\end{description}
		\end{proof}
	},
	sort = Konvergenz
}

\newglossaryentry{Supremum}{
	name = {Supremum},
	description = {
	Sei $M \subset \R$. 
	Eine reelle Zahl $S$ heißt kleinste obere Schranke oder Supremum von $M$, wenn gilt das 
	\begin{enumerate}[1)]
		\item $S$ obere Schranke von $M$ ist, d.h. $\forall x\in M: x \leq S$ und überdies
		\item keine Zahl $< S$ noch obere Schranke von $M$ sein kann, 
			d.h. $\forall \varepsilon > 0 \exists x\in M: x > S - \varepsilon$
	\end{enumerate}
	Ein Supremum ist eindeutig.
	\todoo[Sätze und Beispiele]
	}
}

\newglossaryentry{Infimum}{
	name = {Infimum},
	description = {
	Sei $M \subset \R$. 
	Eine reelle Zahl $s$ heißt größte untere Schranke oder Infimum von $M$, wenn gilt das 
	\begin{enumerate}[1)]
		\item $s$ untere Schranke von $M$ ist, d.h. $\forall x\in M: x \geq s$ und überdies
		\item keine Zahl $> S$ noch untere Schranke von $M$ sein kann, 
			d.h. $\forall \varepsilon > 0 \exists x\in M: x < S - \varepsilon$
	\end{enumerate}
	Ein Infimum ist eindeutig.
	}
}
\newglossaryentry{CF}{
	name = {Cauchy-Folge},
	description = {Sei (X,d) ein metrischer Raum. Eine Folge $\ff{x}{n}{\N}$ in $X$ heißt Cauchy-Folge, wenn gilt
	$$\forall \varepsilon > 0 \exists n \in \N: \forall m \geq  n : |x_m - x_n| < \varepsilon$$
	%
	\begin{itemize}[$\aq$]
		\item $\forall \varepsilon > 0 \exists a \in X, N\in \N: \forall n > N: x_n \in \U[\varepsilon]{a}$
		\item $ d(x_k,x_l) \longrightarrow 0$ für $(k,l) \longrightarrow (\infty, \infty)$
	\end{itemize}
	}
}

\newglossaryentry{vollstandig}{
	name = {vollstandig},
	description = {vollständig}
}

\newglossaryentry{Hilbertraum}{
	name = Hilbertraum,
	description = {vollständiger Skalarproduktvektorraum mit $\norm{\cdot } = \sqrt{(\cdot , \cdot )_X}$.  Wobei $(\cdot , \cdot )$ das Skalarprodukt bezeichnet.}
}

\newglossaryentry{Banachraum}{
	name = Banachraum,
	description = {vollständiger normierter Vektorraum (wir schreiben $(X,\norm{\cdot }_X$)}
}

\newglossaryentry{metrischer Raum}{
	name = {Metrischer Raum},
	description = {Metrischer Raum},
	sort = {Metrischer Raum}
}

\newglossaryentry{Kugel mit Radius r}{
	name = {Kugel mit Radius r},
	description = {
		Sei $(X,d)$ ein \gls{metrischer Raum}. Für $r > 0$ ist die $r$-Umgebung des Punktes x bzw. \textbf{Kugel um $x$ mit Radius $\mathbf{r}$}.
	},
	sort = Metrischer Raum
}

\newglossaryentry{Abschluss}{
	name = {Abschluss},
	description = {$\ov{M}$ Abschluss},
	sort = {Metrischer Raum}
}

\newglossaryentry{Inneres}{
	name = {Inneres},
	description = {$\inner{M}$ Inneres},
	sort = {Metrischer Raum}
}

\newglossaryentry{Rand}{
	name = {Rand},
	description = {$\delta M$ Rand},
	sort = {Metrischer Raum}
}

\newglossaryentry{kompakt}{
	name = {kompakt},
	description = {kompakt}
}

\newglossaryentry{offene Uberdeckung}{
	name = {offene Uberdeckung},
	description = {offene Überdeckung}
}

\newglossaryentry{Stetigkeit}{
	name = {Stetigkeit},
	description = {Stetigkeit}
}

\newglossaryentry{Restklassenbildung}{
	name = {Restklassenbildung},
	description = {Restklassenbildung}
}

\newglossaryentry{Sesquilinearform}{
	name = Sesquilinearform,
	description = {siehe \Gls{Skalarprodukt}},
	sort = Skalarprodukt
}

\newglossaryentry{Hermitesche Form}{
	name = Hermitesche Form,
	description = {siehe \Gls{Skalarprodukt}},
	sort = Skalarprodukt
}

\newglossaryentry{Skalarprodukt}{
	name = Skalarprodukt,
	description = {Skalarprodukt}
}
