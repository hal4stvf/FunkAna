	\chapter{Kompakte Operatoren und Spektraltheorie}
	\section{Kompakte Operatoren}

	\begin{definition}[kompakter Operator]
	\label{def:5.1}
		$X,Y$ normierte Räume. $T: X\to Y$ linearer Operatoren. $T$ heißt \textit{kompakt}, wenn 
		$\ov{T U_1(0)}$ kompakt. 
			$$ K(X,Y) := \set{T: X\to Y, T \text{ kompakt }}, \quad K(X) = K(X,X)$$
	\end{definition}

	\begin{bem}
	\label{bem:5.2}
		%\todog[Muss hier X oder Y z.T. vollständig sein?]	
		\begin{enumerate}[a)]
			\item $K(X,Y) \subset \BS$
			\item Äquivalent sind für $T: X\to Y$, $Y$ Banachraum
				\begin{enumerate}[(i)]
					\item $T\in K(X,Y)$
					\item $M \subset X$ beschränkt $\df$ $T(M)$ präkompakt
					\item $TU_1(0)$ präkompakt
					\item $(x_n)$ beschränkt Folge $\df$ $(Tx_n)$ hat konvergente Teilfolge.
				\end{enumerate}
			\item $I \in K(X) \aq \dim X < \infty$ (Lema von Riesz)
		\end{enumerate}
	\end{bem}
	\begin{proof}
		\todor[Beweis/ Alt S.331]
	\end{proof}

	\begin{bsp}[Integraloperator ist kompakt]
	\label{bsp:5.3}
		$X: C([0,1])$ mit $\norm{\cdot}_\infty$. 
			$$(Tf)(t) = \int_0^t f(\tau)d\tau \df T\in \B(C([0,1]))$$
		Für $f\in C([0,1])$, $\norm{f}_\infty \leq 1$, $t_1,t_2 \in [0,1]$ gilt 
			$$|(Tf)(t_1) - (Tf)(t_2)| =|\int_{t_1}^{t_2} f(\tau) d\tau| \leq |t_2 - t_1|$$
		$\df \set{Tf: \norm{f}_\infty \leq 1}$ ist gleichgeradig stetig. $\df[Arz. Arc]$ präkompakt.
		$\df$ kompakt.
	\end{bsp}

	\begin{thm}[Diverse Aussagen]
	\label{thm:5.4}
		%\todog[Muss hier X oder Y z.T. vollständig sein?]	
		\begin{enumerate}[a)]
			\item $K(X,Y)$ abgeschlossener UVR von $\B(X,Y)$, $Y$ Banachraum.
			\item $T\in K(X,Y)$ ist vollstetig. D.h. $x_n \wkc x \df Tx_n \to Tx$
			\item $X$ reflexiv, $T: X \to Y$ vollstetig $\df$ $T\in K(X,Y)$
			\item $T\in \BS$ mit $rang T < \infty$, dann $T\in K(X,Y)$ (rangT = dim im T)
			\item $Y$ Hilbertraum, $T\in \BS$. Dann $T\in K(X,Y) \aq \exists (T_n) \in \BS$,
							so dass $rang T_n < \infty$ $\forall n\in \N$ und $\norm{T - T_n} \to 0$
			\item $P \in \BS$. Projektor (d.h. $P^2 = P$). Dann $P\in K(X) \aq rang P < \infty$.
			\item $T\in K(X,Y)$, $U\in \B(Y,Z)$, $S \in\B(V,X) \df UTS \in K(V,Z)$
		\end{enumerate}
	\end{thm}
	\begin{proof}[Beweiskizzen]
		\todor[Beweis/ Alt S.332]
	\end{proof}

	\begin{thm}[Satz von Schauder]
	\label{thm:5.5}
		$X,Y$ normierte Räume
			\begin{enumerate}[a)]
				\item $T\in K(X,Y) \df T' \in K(Y',X')$
				\item $T' \in K(Y',X')$, $Y$ vollständig $\df$ $T$ kompakt
			\end{enumerate}
	\end{thm}
	\begin{proof}
		\todor[Beweis. Alt S.405]	
	\end{proof}	

	\begin{definition}[Fredholm-Operator] 	
	\label{def:5.6}
		$A\in \BS$ heißt \emph{Fredholm-Operator}, falls 
			\begin{enumerate}[a)]
				\item $\dim(ker A) < \infty$
				\item $im(A)$ abgeschlossen
				\item $\codim(im A) < \infty$ ($V\subset X$ UVR, $\codim V := \dim X/V)$
			\end{enumerate}
		\emph{Fredholmindex}: $\ind(A) := \dim Ker A - \codim im A$
	\end{definition}

	\begin{thm}[$I-T$ ist ein Fredholm-Operator]
	\label{thm:5.7}
		$T\in K(X)$. Dann ist $A := I - T$ ein Fredholm-Operator mit $\ind(A) = 0$.
	\end{thm}

	\begin{proof}
		\todor	
	\end{proof}

	\begin{bem}[Nichtexistenz eine beschränkten Inversen]
	\label{bem:5.8}
		$X$ normierter Raum, $\dim X = \infty$, $T\in K(X)$. Dann folgt
			$T$ hat keine beschränkte Inverse. \todor[Beweis Übung]
	\end{bem}

	\begin{bem*}[Beispiel]
		$$T:C([0,1]) \to C([0,1]),\; f\mapsto (t\mapsto \int_0^t f(\tau) d\tau)$$
		$(\lambda I - T)f = 0 \df \lambda f(t) = \int_0^t f(\tau) d\tau$, $\lambda = 0 \df f=0$; \{$\lambda \neq 0, f(0) = 0$ und $f'(t) = \frac{1}{\lambda}\}$ $\df f = 0$.
	\end{bem*}

%%%%%%%%%%%%%%%%%%%%%%%%%%%%%%%%%%%%%%%%%%%%%	
	\section{Das Spektrum von beschränkten Operatoren}
%%%%%%%%%%%%%%%%%%%%%%%%%%%%%%%%%%%%%%%%%%%%%	

	\begin{definition}[Das Spektrum]
	\label{def:5.9}
		$T\in \BS[X]$,$X$ Banachraum über $\C$
		\begin{enumerate}[(i)]
		  \item%
				$\rho(T) := \set{\lambda \in \C: ker(\lambda I -T) = \set{0} \text{ und }
				im(\lambda I- T) = X}$%
				\hfill \textbf{\textit{"'Resolventenmenge"'}}%
			\item%
				$\sigma(T) = \C \setminus \rho(T)$%
				\hfill \textbf{\textit{"'Spektrum"'}}
			\item%
				$\sigma_p(T) = \set{\lambda \in \sigma(T):
				ker(\lambda I - T) \neq \set{0}
				}$%
				\hfill \textbf{\textit{"'Punktspektrum"'}}
			\item%
				$\sigma_c(T) = \set{\lambda \in \sigma(T):
				ker(\lambda I - T) =  \set{0} \text{ und }
				im((\lambda I - T)\neq X, \text{aber }
				\ov{im(\lambda I - T)} = X}$%
				\hfill \textbf{\textit{"'kontinuierliches Spektrum"'}}
			\item%
				$\sigma_r(T) = \set{\lambda \in \sigma(T):
				ker(\lambda I - T) = \set{0} \text{ und }
				\ov{im(\lambda I - T)} \neq X}$%
				\hfill \textbf{\textit{"'Residualspektrum"'}}
		\end{enumerate}
	Es gilt $ \sigma(T) = \sigma_p(T) \cup \sigma_r(T) \cup \sigma_c(T)$
	 und wir definieren
	 \begin{enumerate}[(1)]
	   \item	$v\in ker(\lambda I - T) \setminus \set{0}$ heißt \textbf{\textit{ Eigenvektor von $\lambda$ }}.
		 \item $Eig_\lambda(T) := ker(\lambda I - T)$ heißt \textbf{\textit{Eigenraum von $\lambda$}}.
		 \item $\nu_\lambda(T) := dim Eig_\lambda(T)$ heißt \textbf{\textit{geometrische Vielfachheit}}.
		 \item $\mu_\lambda(T) := \max_{n\in\N} \dim ker(\lambda I - T)^n$ heißt
		 	\textbf{\textit{ algebraische Vielfachheit}}
	 \end{enumerate}
	
	\end{definition}

	\begin{thm}[$\rho(T)$ ist offen, Resolventfkt ist Potenzreihe entwickelbar]
	\label{thm:5.10}
		$T\in \BS[X]$, $X$ Banachraum. Dann ist $\rho(T)$ offen und die "'Resolventenfunktion"' 
			$ R(\lambda, T) := \inv{(\lambda I - T)} : \rho(T) \to \BS[X] $ 
		ist komplex-analytisch (d.h. in Potenzreihe entwickelbar).
	\end{thm}

	\begin{proof}
		\todor	
	\end{proof}

	
	\begin{thm}[Spektralradius]
	\label{thm:5.11}
		$X \neq \set{0}$ Banachraum, $T \in \BS[X]$. \\
		$\sigma(T)$ ist kompakt und nichtleer mit "'Spektralradius"':	
					$\disp \sup |\sigma(T)| = \lim_{m\to \infty} \norm{T^m}^{\frac{1}{m}} \leq \norm{T}$.
	\end{thm}
		
	\begin{proof}
		\todor	
	\end{proof}

	\begin{bem}[Diverse Fakten und Beispiele]
	\label{bem:5.12}
			\begin{enumerate}[a)]
				\item $T\in K(X)$, $\lambda \in \sigma(T)\backslash\set{0}$ $\df \lambda \in \sigma_p (T)$. Folgt, da $\lambda I - T = \lambda(I-\frac{T}{\lambda})$
				\item $T\in K(X)$, $\dim X = \infty$ $\df 0 \in \sigma(T)$ (Bem. 5.8)	
				\item $\dim X < \infty \df \sigma(T) = \sigma_p (T)$
				\item Bsp.: Betrachte $T\in \BS[\sigma([0,1])(Tf)(x) = xf(x)$, $\sigma(T) = [0,1] = \sigma_r(T)$ 
				\item $T\in L^p([0,1]), (Tf)(x) = xf(x)$ f.ü. $p < \infty$. $\sigma(T) = [0,1] = \sigma_c(T)$
				\item $L\in \BS[\ell^2]$ Linksshift. $\sigma(T) = \ov{\EK}$
					$\sigma_p(T) = \EK$; $\sigma_c(T) = \partial \EK $
				\item $R = L^* \in \BS[\ell^2]$ Rechtsshift. $\sigma(T) = \ov{\EK}$, $\sigma_r(T) = \EK$, $\sigma_c(T) = \partial \EK$.
			\end{enumerate}
	\end{bem}					


	\begin{thm}[Spektralsatz für kompakte Operatoren]
	\label{thm:5.13}
		$X$ Banachraum, $T\in K(X)$. Dann gilt
			\begin{enumerate}[a)]
				\item $\sigma(T)\backslash\set{0}$ aus höchstens abzählbar vielen Eigenwerten mit $0$ als einzig möglichen Häufungspunkt. Wenn $\sigma(T)$ unendlich, dann ist $0\in\sigma(T)$.
				\item Für $\lambda \in \sigma(T)\backslash\set{0}$ ist 
					$1\leq n_\lambda := \max\set{n\in \N: ker(\lambda I - T)^{n-1} \neq ker(\lambda I-T)^n} < \infty$
					\item \enquote{Riesz-Zerlegung}. Für $\lambda \in \sigma(T)\backslash\set{0}$ gilt 
					$X = ker(\lambda I - T)^{n_\lambda} \oplus im(\lambda I - T)^{n_\lambda}$ 
	 				\item $\lambda \in \sigma(T)\backslash \set{0} \df \sigma(T| _{im(\lambda I - T)^{n_\lambda}}) = \sigma(T)\backslash \set{\lambda}$
					\item Ist $E_\lambda \in \BS[X]$ Projektor mit $ker E_\lambda = im(\lambda I - T)^{n_\lambda}$, $im E_\lambda = ker(\lambda I - T)^{n_\lambda}$, so gilt $E_{\lambda} E_{\mu} = \delta_{\lambda \mu} E_\lambda $
			\end{enumerate}
	\end{thm}
	\begin{proof}
		\todor[Beweis. Alt S.395]	
	\end{proof}	

	
	Hier fehlt nichts!!!!!
	\setcounter{thm}{17}
	\begin{cor}[Fredholm Alternative]
	\label{thm:5.18}
		$T\in K(X)$, $\lambda \neq 0$. Dann gilt: 
			\begin{itemize}[]
				\item Entweder ist die Gleichung $Tx - \lambda x = y$ nach $x$ eindeutig lösbar $\forall y\in X$
				\item oder $Tx - \lambda x = 0$ hat nicht-triviale Lösungen.
			\end{itemize}
	\end{cor}
	\begin{proof}
		\todor[Beweis. Alt S.407]	
	\end{proof}	

	\begin{bem}[Beispiel]
	\label{bem:5.19}
		Operatoren ohne Eigenwerte: $T\in K(C[0,1])$, wie in Bsp. 5.3.
			$$ (Tf)(t) = \int_0^t f(\tau)d\tau$$
	\end{bem}

	%%%%%%%%%%%%%%%%%%%%%%%%%%%%%%%%%%%%%%%%%%%%%%%%%%%%%%%%%%%%%%%%%%%%%%%%%%%%
	\section{Kompakte normale/selbstadjungierte Operatoren}
	%%%%%%%%%%%%%%%%%%%%%%%%%%%%%%%%%%%%%%%%%%%%%%%%%%%%%%%%%%%%%%%%%%%%%%%%%%%%
	Sei $X$ von nun an ein Hilbertraum und meistens $\K = \C$

	\begin{definition}[selbstadjungiert, normal]
	\label{def:5.20}
		$A \in \BS[X]$. Dann heißt $A$ 
			\begin{enumerate}[a)]
				\item selbstadjungiert, wenn $A = A^*$
				\item normal, falls $A^* A = A A^* \aq[{\hyperref[B10.3]{UE10}}] 
					\norm{Ax} = \norm{A^* x} \; \forall x \in X$ 
			\end{enumerate}
	\end{definition}
	
	\begin{lemma}[Norm über größtes Spektrum]
	\label{lem:5.21}
		$X \neq \set{0}$ $\C$-HR, $T\in \BS[X]$ normal. Dann ist $\sup|\sigma(T)| = \norm{T}$.
	\end{lemma}

	\begin{proof}
		\todor	
	\end{proof}

		
	\begin{thm}[Spektralsatz für kompakte und normale Operatoren]
	\label{thm:5.22}
		$X$ $\C$-HR, $T\in K(X)\backslash\set{0}$ normal. Dann gilt:
			\begin{enumerate}[a)]
				\item $\exists N\subset \N$, ONS $\ff{e}{n}{N} \in X$, 
					$\ff{\lambda}{n}{N}\in \C\backslash\set{0},$ s.d. 
					$$ T e_k = \lambda_k e_k \quad \forall k\in N, \quad 
						\sigma(T)\backslash\set{0} = \set{\lambda_k: k \in N} $$
				Ist $N = \N$, so gilt $\disp \limes \lambda_n = 0$
				\item $n_{\lambda_k} = 1 \quad \forall k\in\N$
				\item $X = ker T \perp span\set{e_n : n\in N}$
				\item $Tx = \sum_{k\in N} \lambda_k \Sp{x,e_k} e_k \quad \forall x \in X$
			\end{enumerate}
	\end{thm}
	
	\begin{proof}
		\todor	
	\end{proof}

	\begin{bem}[Diverse Fakten]
	\label{bem:5.23}
	  $X$ $\C$-HR, $T\in\BS[X]$	
			\begin{enumerate}[a)]
				\item $T=T^*, \sigma(T) \subset [-\norm{T},\norm{T}]$
					($\sigma(T)$ ist eine Teilmenge von $\R$, wenn $T$ selbstadjungiert ist.
					Außerdem ist $\sigma(T) \subset \ov{B_{\norm{T}}(0)}$)
				\item $T = T^*$, $T\in K(X)$ $\df$ $\norm{T} \in \sigma_p(T) \vee
					-\norm{T} \in \sigma_p(T)$ Lemma 5.23 und 5.22)
				\item $T = T^*$, $T$ positiv semi-definit, d.h. $(Tx,x) \geq 0 \forall x$
					$ \df \sigma(T) \subset [0, \norm{T}]$ 
					(UE: z.z. $\forall x < 0 \df \lambda \in \rho(T)$)
				\item $T = T^*$, $T\in K(X)$ pos. semidefinit $\df$
					$\norm{T} \in \rho_p(T)$ (aus b) und c))
				\item $T = T^*$, $T\in K(X)$ pos. semidefinit $\df$
					$\exists \; \sqrt{T} \in K(X)$, s.d 
					$\sqrt{T} = \sqrt{T^*} (\sqrt{T})^*$ und $(\sqrt{T})^2 = T$
				\item $T\in K(X,Y), Y \C$-HR $\df$
					$\exists \; |T|\in K(X)$, so dass $|T|$ pos. semidefinit, 
					selbstadjungiert und $T^* T = |T|^2$
			\end{enumerate}
	\end{bem}
	
		\todor[kurz begründen]
	\begin{proof}
		\begin{enumerate}[a)]
		  \item 
		  \item 
		  \item 
		  \item 
		  \item \todor[kleine Rechnung]
		  \item	$T^* T \in K(X)$, pos. semidef. $\df \exists \sqrt{T^* T}$ wie in e). 
				Definiere $|T| = \sqrt{T^* T}$
		\end{enumerate}
	\end{proof}
